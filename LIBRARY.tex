%% CHAPTER 1


@article{Etard2020,
abstract = {Aim: Trait data are increasingly being used in studies investigating the impacts of global changes on the structure and functioning of ecological communities. Despite a growing number of trait data collations for terrestrial vertebrates, there is to date no global assessment of the gaps and biases the data present. Here, we assess whether terrestrial vertebrate trait data are taxonomically, spatially and phylogenetically biased. Location: Global. Time period: Present. Major taxa studied: Terrestrial vertebrates. Methods: We compile seven ecological traits and quantify coverage as the proportion of species for which an estimate is available. For a species, we define completeness as the proportion of non-missing values across traits. We assess whether coverage and completeness differ across classes and examine phylogenetic biases in trait data. To investigate spatial biases, we test whether wider-ranging species have more complete trait data than narrow-ranging species. Additionally, we test whether species-rich regions, which are of most concern for conservation, are less well sampled than species-poor regions. Results: Mammals and birds are well sampled even in species-rich regions. For reptiles and amphibians (herptiles), only body size presents a high coverage (>80%), in addition to habitat-related variables (amphibians). Herptiles are poorly sampled for other traits. The shortfalls are particularly acute in some species-rich regions and for certain clades. Across all classes, geographically rarer species have less complete trait information. Main conclusions: Trait information is less available on average in some of the most diverse areas and in geographically rarer species, both of which crucial for biodiversity conservation. Gaps in trait data might impede our ability to conduct large-scale analyses, whereas biases can impact the validity of extrapolations. A short-term solution to the problem is to estimate missing trait data using imputation techniques, whereas a longer-term and more robust filling of existing gaps requires continued data-collection efforts.},
author = {Etard, Adrienne and Morrill, Sophie and Newbold, Tim},
doi = {10.1111/geb.13184},
file = {:C\:/Users/adrie/AppData/Local/Mendeley Ltd./Mendeley Desktop/Downloaded/Etard, Morrill, Newbold - 2020 - Global gaps in trait data for terrestrial vertebrates.pdf:pdf},
issn = {14668238},
journal = {Global Ecology and Biogeography},
keywords = {completeness,coverage,phylogenetic biases,spatial biases,taxonomic biases,terrestrial vertebrates,traits},
number = {November 2019},
pages = {1--16},
title = {{Global gaps in trait data for terrestrial vertebrates}},
year = {2020}
}

@article{Escudero2016,
abstract = {Functional traits are the center of recent attempts to unify key ecological theories on species coexistence and assembling in populations and communities. While the plethora of studies on the role of functional traits to explain patterns and dynamics of communities has rendered a complex picture due to the idiosyncrasies of each study system and approach, there is increasing evidence on their actual relevance when aspects such as different spatial scales, intraspecific variability and demography are considered.},
author = {Escudero, Adri{\'{a}}n and Valladares, Fernando},
booktitle = {Oecologia},
doi = {10.1007/s00442-016-3578-5},
file = {:C\:/Users/adrie/Downloads/Escudero-Valladares2016_Article_Trait-basedPlantEcologyMovingT.pdf:pdf},
issn = {00298549},
pmid = {26897604},
title = {{Trait-based plant ecology: moving towards a unifying species coexistence theory: Features of the Special Section}},
year = {2016}
}

@article {Zamudio2016,
	author = {Zamudio, Kelly R. and Bell, Rayna C. and Mason, Nicholas A.},
	title = {Phenotypes in phylogeography: Species{\textquoteright} traits, environmental variation, and vertebrate diversification},
	volume = {113},
	number = {29},
	pages = {8041--8048},
	year = {2016},
	doi = {10.1073/pnas.1602237113},
	publisher = {National Academy of Sciences},
	abstract = {Almost 30 y ago, the field of intraspecific phylogeography laid the foundation for spatially explicit and genealogically informed studies of population divergence. With new methods and markers, the focus in phylogeography shifted to previously unrecognized geographic genetic variation, thus reducing the attention paid to phenotypic variation in those same diverging lineages. Although phenotypic differences among lineages once provided the main data for studies of evolutionary change, the mechanisms shaping phenotypic differentiation and their integration with intraspecific genetic structure have been underexplored in phylogeographic studies. However, phenotypes are targets of selection and play important roles in species performance, recognition, and diversification. Here, we focus on three questions. First, how can phenotypes elucidate mechanisms underlying concordant or idiosyncratic responses of vertebrate species evolving in shared landscapes? Second, what mechanisms underlie the concordance or discordance of phenotypic and phylogeographic differentiation? Third, how can phylogeography contribute to our understanding of functional phenotypic evolution? We demonstrate that the integration of phenotypic data extends the reach of phylogeography to explain the origin and maintenance of biodiversity. Finally, we stress the importance of natural history collections as sources of high-quality phenotypic data that span temporal and spatial axes.},
	issn = {0027-8424},
	URL = {https://www.pnas.org/content/113/29/8041},
	eprint = {https://www.pnas.org/content/113/29/8041.full.pdf},
	journal = {Proceedings of the National Academy of Sciences}
}


@article{Diaz2016,
abstract = {Earth is home to a remarkable diversity of plant forms and life histories, yet comparatively few essential trait combinations have proved evolutionarily viable in today's terrestrial biosphere. By analysing worldwide variation in six major traits critical to growth, survival and reproduction within the largest sample of vascular plant species ever compiled, we found that occupancy of six-dimensional trait space is strongly concentrated, indicating coordination and trade-offs. Three- quarters of trait variation is captured in a two-dimensional global spectrum of plant form and function. One major dimension within this plane reflects the size of whole plants and their parts; the other represents the leaf economics spectrum, which balances leaf construction costs against growth potential. The global plant trait spectrum provides a backdrop for elucidating constraints on evolution, for functionally qualifying species and ecosystems, and for improving models that predict future vegetation based on continuous variation in plant form and function.},
author = {D{\'{i}}az, Sandra and Kattge, Jens and Cornelissen, Johannes H.C. and Wright, Ian J. and Lavorel, Sandra and Dray, St{\'{e}}phane and Reu, Bj{\"{o}}rn and Kleyer, Michael and Wirth, Christian and {Colin Prentice}, I. and Garnier, Eric and B{\"{o}}nisch, Gerhard and Westoby, Mark and Poorter, Hendrik and Reich, Peter B. and Moles, Angela T. and Dickie, John and Gillison, Andrew N. and Zanne, Amy E. and Chave, J{\'{e}}r{\^{o}}me and {Joseph Wright}, S. and {Sheremet Ev}, Serge N. and Jactel, Herv{\'{e}} and Baraloto, Christopher and Cerabolini, Bruno and Pierce, Simon and Shipley, Bill and Kirkup, Donald and Casanoves, Fernando and Joswig, Julia S. and G{\"{u}}nther, Angela and Falczuk, Valeria and R{\"{u}}ger, Nadja and Mahecha, Miguel D. and Gorn{\'{e}}, Lucas D.},
doi = {10.1038/nature16489},
file = {:C\:/Users/adrie/Downloads/nature16489.pdf:pdf},
issn = {14764687},
journal = {Nature},
mendeley-groups = {UCL PhD},
title = {{The global spectrum of plant form and function}},
year = {2016}
}

@article{Bohm2016,
abstract = {One-fifth of the world's reptiles are currently estimated as threatened with extinction, primarily due to the immediate threats of habitat loss and overexploitation. Climate change presents an emerging slow-acting threat. However, few IUCN Red List assessments for reptiles explicitly consider the potential role of climate change as a threat. Thus, climate change vulnerability assessments can complement existing Red List assessments and highlight further, emerging priorities for conservation action. Here we present the first trait-based global climate change vulnerability assessment for reptiles to estimate the climate change vulnerability of a random representative sample of 1498 species of reptiles. We collected species-specific traits relating to three dimensions of climate change, sensitivity, low adaptability, and exposure, which we combined to assess overall vulnerability. We found 80.5% of species highly sensitive to climate change, primarily due to habitat specialisation, while 48% had low adaptability and 58% had high exposure. Overall, 22% of species assessed were highly vulnerable to climate change. Hotspots of climate change vulnerability did not always overlap with hotspots of threatened species richness, with most of the vulnerable species found in northwestern South America, southwestern USA, Sri Lanka, the Himalayan Arc, Central Asia and southern India. Most families were found to be significantly more vulnerable to climate change than expected by chance. Our findings build on previous work on reptile extinction risk to provide an overview of the risk posed to reptiles by climate change. Despite significant data gaps for a number of traits, we recommend that these findings are integrated into reassessments of species' extinction risk, to monitor both immediate and slow-acting threats to reptiles.},
author = {Bohm, Monika and Cook, Daniel and Ma, Heidi and Davidson, Ana D. and Garc{\"{i}}¿½a, Andr{\"{i}}¿½s and Tapley, Benjamin and Pearce-Kelly, Paul and Carr, Jamie},
doi = {10.1016/j.biocon.2016.06.002},
isbn = {0006-3207},
issn = {00063207},
journal = {Biological Conservation},
keywords = {Adaptability,Climate change,Exposure,Herpetology,IUCN Red List,Sensitivity},
mendeley-groups = {UCL PhD/Papers},
title = {{Hot and bothered: Using trait-based approaches to assess climate change vulnerability in reptiles}},
year = {2016}
}


@misc{Pacifici2015,
abstract = {The effects of climate change on biodiversity are increasingly well documented, and many methods have been developed to assess species' vulnerability to climatic changes, both ongoing and projected in the coming decades. To minimize global biodiversity losses, conservationists need to identify those species that are likely to be most vulnerable to the impacts of climate change. In this Review, we summarize different currencies used for assessing species' climate change vulnerability. We describe three main approaches used to derive these currencies (correlative, mechanistic and trait-based), and their associated data requirements, spatial and temporal scales of application and modelling methods. We identify strengths and weaknesses of the approaches and highlight the sources of uncertainty inherent in each method that limit projection reliability. Finally, we provide guidance for conservation practitioners in selecting the most appropriate approach(es) for their planning needs and highlight priority areas for further assessments.},
archivePrefix = {arXiv},
arxivId = {arXiv:0811.2183v2},
author = {Pacifici, Michela and Foden, Wendy B. and Visconti, Piero and Watson, James E.M. and Butchart, Stuart H.M. and Kovacs, Kit M. and Scheffers, Brett R. and Hole, David G. and Martin, Tara G. and Ak{\c{c}}akaya, H. Resit and Corlett, Richard T. and Huntley, Brian and Bickford, David and Carr, Jamie A. and Hoffmann, Ary A. and Midgley, Guy F. and Pearce-Kelly, Paul and Pearson, Richard G. and Williams, Stephen E. and Willis, Stephen G. and Young, Bruce and Rondinini, Carlo},
booktitle = {Nature Climate Change},
doi = {10.1038/nclimate2448},
eprint = {arXiv:0811.2183v2},
isbn = {1758-678X},
issn = {17586798},
mendeley-groups = {UCL PhD/Papers,UCL PhD},
pmid = {23347591},
title = {{Assessing species vulnerability to climate change}},
year = {2015}
}

@article{Pearson2014,
abstract = {There is an urgent need to develop effective vulnerability assessments for evaluating the conservation status of species in a changing climate1 . Several new assessment approaches have been proposed for evaluating the vulnerability of species to climate change2–5 based on the expectation that established assessments such as the IUCN Red List6 need revising or superseding in light of the threat that climate change brings. However, although previous studies have identified ecological andlife history attributes that characterize declining species or those listed as threatened7–9 , no study so far has undertaken a quantitative analysis of the attributes that cause species to be at high risk of extinction specifically due to climate change.We developed a simulation approach based on generic life history types to show here that extinction risk due to climate change can be predicted using a mixture of spatial and demographic variables that can be measured in the present day without the need for complex forecasting models. Most of the variableswe found to be important for predicting extinction risk, including occupied area and population size, are already used in species conservation assessments, indicating that present systems may be better able to identify species vulnerable to climate change than previously thought. Therefore, although climate change brings many newconservation challenges,we find that it may not be fundamentally different from other threats in terms of assessing extinction risks.},
author = {Pearson, Richard G. and Stanton, Jessica C. and Shoemaker, Kevin T. and Aiello-Lammens, Matthew E. and Ersts, Peter J. and Horning, Ned and Fordham, Damien A. and Raxworthy, Christopher J. and Ryu, Hae Yeong and Mcnees, Jason and Ak{\c{c}}akaya, H. Reşit},
doi = {10.1038/nclimate2113},
isbn = {1758-678X},
issn = {17586798},
journal = {Nature Climate Change},
mendeley-groups = {UCL PhD/Papers,Project outline},
pmid = {100251034},
title = {{Life history and spatial traits predict extinction risk due to climate change}},
year = {2014}
}

@article{Newbold2013,
abstract = {Land-use change is one of the main drivers of current and likely future biodiversity loss. Therefore, understanding how species are affected by it is crucial to guide conservation decisions. Species respond differently to land-use change, possibly related to their traits. Using pan-tropical data on bird occurrence and abundance across a human land-use intensity gradient, we tested the effects of seven traits on observed responses. A likelihood-based approach allowed us to quantify uncertainty in modelled responses, essential for applying the model to project future change. Compared with undisturbed habitats, the average probability of occurrence of bird species was 7.8 per cent and 31.4 per cent lower, and abundance declined by 3.7 per cent and 19.2 per cent in habitats with low and high human land-use intensity, respectively. Five of the seven traits tested affected the observed responses significantly: long-lived, large, non-migratory, primarily frugivorous or insectivorous forest specialists were both less likely to occur and less abundant in more intensively used habitats than short-lived, small, migratory, non-frugivorous/insectivorous habitat generalists. The finding that species responses to land use depend on their traits is important for understanding ecosystem functioning, because species' traits determine their contribution to ecosystem processes. Furthermore, the loss of species with particular traits might have implications for the delivery of ecosystem services.},
author = {Newbold, Tim and Scharlemann, J{\"{o}}rn P W and Butchart, Stuart H M and Sekercioğlu, Cağan H and Alkemade, Rob and Booth, Hollie and Purves, Drew W},
doi = {10.1098/rspb.2012.2131},
isbn = {0962-8452, 1471-2954},
issn = {1471-2954},
journal = {Proceedings. Biological sciences / The Royal Society},
keywords = {Animals,Biodiversity,Birds,Birds: physiology,Environment,Human Activities,Humans,Likelihood Functions,Models, Biological,Tropical Climate},
mendeley-groups = {UCL PhD/Papers,Project outline,UCL PhD},
pmid = {23173205},
title = {{Ecological traits affect the response of tropical forest bird species to land-use intensity.}},
year = {2013}
}

@misc{Lavorel2002a,
abstract = {1. The concept of plant functional type proposes that species can be grouped accord- ing to common responses to the environment and/or common effects on ecosystem processes. However, the knowledge of relationships between traits associated with the response of plants to environmental factors such as resources and disturbances (response traits), and traits that determine effects of plants on ecosystem functions (effect traits), such as biogeochemical cycling or propensity to disturbance, remains rudimentary. 2. We present a framework using concepts and results from community ecology, ecosystem ecology and evolutionary biology to provide this linkage. Ecosystem func- tioning is the end result of the operation of multiple environmental filters in a hierarchy of scales which, by selecting individuals with appropriate responses, result in assem- blages with varying trait composition. Functional linkages and trade-offs among traits, each of which relates to one or several processes, determine whether or not filtering by different factors gives a match, and whether ecosystem effects can be easily deduced from the knowledge of the filters. 3. To illustrate this framework we analyse a set of key environmental factors and ecosystem processes. While traits associated with response to nutrient gradients strongly overlapped with those determining net primary production, little direct overlap was found between response to fire and flammability. 4. We hypothesize that these patterns reflect general trends. Responses to resource availability would be determined by traits that are also involved in biogeochemical cycling, because both these responses and effects are driven by the trade-off between acquisition and conservation. On the other hand, regeneration and demographic traits associated with response to disturbance, which are known to have little connection with adult traits involved in plant ecophysiology, would be of little relevance to ecosystem processes. 5. This framework is likely to be broadly applicable, although caution must be exer- cised to use trait linkages and trade-offs appropriate to the scale, environmental con- ditions and evolutionary context. It may direct the selection of plant functional types for vegetation models at a range of scales, and help with the design of experimental studies of relationships between plant diversity and ecosystem properties.},
author = {Lavorel, S. and Garnier, E.},
booktitle = {Functional Ecology},
doi = {10.1046/j.1365-2435.2002.00664.x},
isbn = {0269-8463},
issn = {02698463},
keywords = {Biogeochemical cycles,Disturbance,Effect traits,Fire,Resource gradient,Response traits,Scaling-up,Trade-offs},
mendeley-groups = {UCL PhD/Papers},
pmid = {2026},
title = {{Predicting changes in community composition and ecosystem functioning from plant traits: Revisiting the Holy Grail}},
year = {2002}
}

@misc{Violle2007,
abstract = {In its simplest definition, a trait is a surrogate of organismal performance, and this meaning of the term has been used by evolutionists for a long time. Over the last three decades, developments in community and ecosystem ecology have forced the concept of trait beyond these original boundaries, and trait-based approaches are now widely used in studies ranging from the level of organisms to that of ecosystems. Despite some attempts to fix the terminology, especially in plant ecology, there is currently a high degree of confusion in the use, not only of the term “trait” itself, but also in the underlying concepts it refers to. We therefore give an unambiguous definition of plant trait, with a particular emphasis on functional trait. A hierarchical perspective is proposed, extending the “performance paradigm” to plant ecology. “Functional traits” are defined as morpho-physio-phenological traits which impact fitness indirectly via their effects on growth, reproduction and survival, the three components of individual performance. We finally present an integrative framework explaining how changes in trait values due to environmental variations are translated into organismal performance, and how these changes may influence processes at higher organizational levels. We argue that this can be achieved by developing “integration functions” which can be grouped into functional response (community level) and effect (ecosystem level) algorithms.},
author = {Violle, Cyrille and Navas, Marie Laure and Vile, Denis and Kazakou, Elena and Fortunel, Claire and Hummel, Ir{\`{e}}ne and Garnier, Eric},
booktitle = {Oikos},
doi = {10.1111/j.0030-1299.2007.15559.x},
isbn = {0030-1299},
issn = {16000706},
mendeley-groups = {UCL PhD/Papers,UCL PhD},
pmid = {245814500017},
title = {{Let the concept of trait be functional!}},
year = {2007}
}

@misc{Wong2018,
abstract = {In focusing on how organisms' generalizable functional properties (traits) interact mechanistically with environments across spatial scales and levels of biological organization, trait-based approaches provide a powerful framework for attaining synthesis, generality and prediction. Trait-based research has considerably improved understanding of the assembly, structure and functioning of plant communities. Further advances in ecology may be achieved by exploring the trait-environment relationships of non-sessile, heterotrophic organisms such as terrestrial arthropods, which are geographically ubiquitous, ecologically diverse, and often important functional components of ecosystems. Trait-based studies and trait databases have recently been compiled for groups such as ants, bees, beetles, butterflies, spiders and many others; however, the explicit justification, conceptual framework, and primary-evidence base for the burgeoning field of 'terrestrial arthropod trait-based ecology' have not been well established. Consequently, there is some confusion over the scope and relevance of this field, as well as a tendency for studies to overlook important assumptions of the trait-based approach. Here we aim to provide a broad and accessible overview of the trait-based ecology of terrestrial arthropods. We first define and illustrate foundational concepts in trait-based ecology with respect to terrestrial arthropods, and justify the application of trait-based approaches to the study of their ecology. Next, we review studies in community ecology where trait-based approaches have been used to elucidate how assembly processes for terrestrial arthropod communities are influenced by niche filtering along environmental gradients (e.g. climatic, structural, and land-use gradients) and by abiotic and biotic disturbances (e.g. fire, floods, and biological invasions). We also review studies in ecosystem ecology where trait-based approaches have been used to investigate biodiversity-ecosystem function relationships: how the functional diversity of arthropod communities relates to a host of ecosystem functions and services that they mediate, such as decomposition, pollination and predation. We then suggest how future work can address fundamental assumptions and limitations by investigating trait functionality and the effects of intraspecific variation, assessing the potential for sampling methods to bias the traits and trait values observed, and enhancing the quality and consolidation of trait information in databases. A roadmap to guide observational trait-based studies is also presented. Lastly, we highlight new areas where trait-based studies on terrestrial arthropods are well positioned to advance ecological understanding and application. These include examining the roles of competitive, non-competitive and (multi-)trophic interactions in shaping coexistence, and macro-scaling trait-environment relationships to explain and predict patterns in biodiversity and ecosystem functions across space and time. We hope this review will spur and guide future applications of the trait-based framework to advance ecological insights from the most diverse eukaryotic organisms on Earth.},
author = {Wong, Mark K.L. and Gu{\'{e}}nard, Benoit and Lewis, Owen T.},
booktitle = {Biological Reviews},
doi = {10.1111/brv.12488},
issn = {1469185X},
keywords = {ant,bee,beetle,butterfly,community assembly,ecosystem function,functional diversity,functional trait,insect,invertebrate,review,spider},
mendeley-groups = {UCL PhD/Papers},
title = {{Trait-based ecology of terrestrial arthropods}},
year = {2018}
}

@article{Kattge2011,
abstract = {Plant traits - the morphological, anatomical, physiological, biochemical and phenological characteristics of plants and their organs - determine how primary producers respond to environmental factors, affect other trophic levels, influence ecosystem processes and services and provide a link from species richness to ecosystem functional diversity. Trait data thus represent the raw material for a wide range of research from evolutionary biology, community and functional ecology to biogeography. Here we present the global database initiative named TRY, which has united a wide range of the plant trait research community worldwide and gained an unprecedented buy-in of trait data: so far 93 trait databases have been contributed. The data repository currently contains almost three million trait entries for 69000 out of the world's 300000 plant species, with a focus on 52 groups of traits characterizing the vegetative and regeneration stages of the plant life cycle, including growth, dispersal, establishment and persistence. A first data analysis shows that most plant traits are approximately log-normally distributed, with widely differing ranges of variation across traits. Most trait variation is between species (interspecific), but significant intraspecific variation is also documented, up to 40% of the overall variation. Plant functional types (PFTs), as commonly used in vegetation models, capture a substantial fraction of the observed variation - but for several traits most variation occurs within PFTs, up to 75% of the overall variation. In the context of vegetation models these traits would better be represented by state variables rather than fixed parameter values. The improved availability of plant trait data in the unified global database is expected to support a paradigm shift from species to trait-based ecology, offer new opportunities for synthetic plant trait research and enable a more realistic and empirically grounded representation of terrestrial vegetation in Earth system models. {\textcopyright} 2011 Blackwell Publishing Ltd.},
author = {Kattge, J. and D{\'{i}}az, S. and Lavorel, S. and Prentice, I. C. and Leadley, P. and B{\"{o}}nisch, G. and Garnier, E. and Westoby, M. and Reich, P. B. and Wright, I. J. and Cornelissen, J. H.C. and Violle, C. and Harrison, S. P. and {Van Bodegom}, P. M. and Reichstein, M. and Enquist, B. J. and Soudzilovskaia, N. A. and Ackerly, D. D. and Anand, M. and Atkin, O. and Bahn, M. and Baker, T. R. and Baldocchi, D. and Bekker, R. and Blanco, C. C. and Blonder, B. and Bond, W. J. and Bradstock, R. and Bunker, D. E. and Casanoves, F. and Cavender-Bares, J. and Chambers, J. Q. and Chapin, F. S. and Chave, J. and Coomes, D. and Cornwell, W. K. and Craine, J. M. and Dobrin, B. H. and Duarte, L. and Durka, W. and Elser, J. and Esser, G. and Estiarte, M. and Fagan, W. F. and Fang, J. and Fern{\'{a}}ndez-M{\'{e}}ndez, F. and Fidelis, A. and Finegan, B. and Flores, O. and Ford, H. and Frank, D. and Freschet, G. T. and Fyllas, N. M. and Gallagher, R. V. and Green, W. A. and Gutierrez, A. G. and Hickler, T. and Higgins, S. I. and Hodgson, J. G. and Jalili, A. and Jansen, S. and Joly, C. A. and Kerkhoff, A. J. and Kirkup, D. and Kitajima, K. and Kleyer, M. and Klotz, S. and Knops, J. M.H. and Kramer, K. and K{\"{u}}hn, I. and Kurokawa, H. and Laughlin, D. and Lee, T. D. and Leishman, M. and Lens, F. and Lenz, T. and Lewis, S. L. and Lloyd, J. and Llusi{\`{a}}, J. and Louault, F. and Ma, S. and Mahecha, M. D. and Manning, P. and Massad, T. and Medlyn, B. E. and Messier, J. and Moles, A. T. and M{\"{u}}ller, S. C. and Nadrowski, K. and Naeem, S. and Niinemets, {\"{U}} and N{\"{o}}llert, S. and N{\"{u}}ske, A. and Ogaya, R. and Oleksyn, J. and Onipchenko, V. G. and Onoda, Y. and Ordo{\~{n}}ez, J. and Overbeck, G. and Ozinga, W. A. and Pati{\~{n}}o, S. and Paula, S. and Pausas, J. G. and Pe{\~{n}}uelas, J. and Phillips, O. L. and Pillar, V. and Poorter, H. and Poorter, L. and Poschlod, P. and Prinzing, A. and Proulx, R. and Rammig, A. and Reinsch, S. and Reu, B. and Sack, L. and Salgado-Negret, B. and Sardans, J. and Shiodera, S. and Shipley, B. and Siefert, A. and Sosinski, E. and Soussana, J. F. and Swaine, E. and Swenson, N. and Thompson, K. and Thornton, P. and Waldram, M. and Weiher, E. and White, M. and White, S. and Wright, S. J. and Yguel, B. and Zaehle, S. and Zanne, A. E. and Wirth, C.},
doi = {10.1111/j.1365-2486.2011.02451.x},
issn = {13652486},
journal = {Global Change Biology},
keywords = {Comparative ecology,Database,Environmental gradient,Functional diversity,Global analysis,Global change,Interspecific variation,Intraspecific variation,Plant attribute,Plant functional type,Plant trait,Vegetation model},
title = {{TRY - a global database of plant traits}},
year = {2011}
}

@article{Titley2017,
abstract = {Over the last 25 years, research on biodiversity has expanded dramatically, fuelled by increasing threats to the natural world. However, the number of published studies is heavily weighted towards certain taxa, perhaps influencing conservation awareness of and funding for less-popular groups. Few studies have systematically quantified these biases, although information on this topic is important for informing future research and conservation priorities. We investigated: i) which animal taxa are being studied; ii) if any taxonomic biases are the same in temperate and tropical regions; iii) whether the taxon studied is named in the title of papers on biodiversity, perhaps reflecting a perception of what biodiversity is; iv) the geographical distribution of biodiversity research, compared with the distribution of biodiversity and threatened species; and v) the geographical distribution of authors' countries of origin. To do this, we used the search engine Web of Science to systematically sample a subset of the published literature with ‘biodiversity' in the title. In total 526 research papers were screened—5% of all papers in Web of Science with biodiversity in the title. For each paper, details on taxonomic group, title phrasing, number of citations, study location, and author locations were recorded. Compared to the proportions of described species, we identified a considerable taxonomic weighting towards vertebrates and an under-representation of invertebrates (particularly arachnids and insects) in the published literature. This discrepancy is more pronounced in highly cited papers, and in tropical regions, with only 43% of biodiversity research in the tropics including invertebrates. Furthermore, while papers on vertebrate taxa typically did not specify the taxonomic group in the title, the converse was true for invertebrate papers. Biodiversity research is also biased geographically: studies are more frequently carried out in developed countries with larger economies, and for a given level of species or threatened species, tropical countries were understudied relative to temperate countries. Finally, biodiversity research is disproportionately authored by researchers from wealthier countries, with studies less likely to be carried out by scientists in lower-GDP nations. Our results highlight the need for a more systematic and directed evaluation of biodiversity studies, perhaps informing more targeted research towards those areas and taxa most depauperate in research. Only by doing so can we ensure that biodiversity research yields results that are relevant and applicable to all regions and that the information necessary for the conservation of threatened species is available to conservation practitioners.},
author = {Titley, Mark A. and Snaddon, Jake L. and Turner, Edgar C.},
doi = {10.1371/journal.pone.0189577},
isbn = {1111111111},
issn = {19326203},
journal = {PLoS ONE},
mendeley-groups = {UCL PhD/Papers,Project outline},
title = {{Scientific research on animal biodiversity is systematically biased towards vertebrates and temperate regions}},
year = {2017}
}

@article{Galan-Acedo2019,
abstract = {Ecosystems largely depend, for both their functioning and their ecological integrity, on the ecological traits of the species that inhabit them. Non-human primates have a wide geographic distribution and play vital roles in ecosystem structure, function, and resilience. However, there is no comprehensive and updated compilation of information on ecological traits of all the world's primate species to accurately assess such roles at a global scale. Here we present a database on some important ecological traits of the world's primates (504 species), including home range size, locomotion type, diel activity, trophic guild, body mass, habitat type, current conservation status, population trend, and geographic realm. We compiled this information through a careful review of 1,216 studies published between 1941 and 2018, resulting in a comprehensive, easily accessible and user-friendly database. This database has broad applicability in primatological studies, and can potentially be used to address many research questions at all spatial scales, from local to global.},
author = {Gal{\'{a}}n-Acedo, Carmen and Arroyo-Rodr{\'{i}}guez, V{\'{i}}ctor and Andresen, Ellen and Arasa-Gisbert, Ricard},
doi = {10.1038/s41597-019-0059-9},
issn = {20524463},
journal = {Scientific data},
mendeley-groups = {UCL PhD},
title = {{Ecological traits of the world's primates}},
year = {2019}
}

@article{Jones2009,
abstract = {Analyses of life-history, ecological, and geographic trait differences among species, their causes, correlates, and likely consequences are increasingly important for understanding and conserving biodiversity in the face of rapid global change. Assembling multispecies trait data from diverse literature sources into a single comprehensive data set requires detailed consideration of methods to reliably compile data for particular species, and to derive single estimates from multiple sources based on different techniques and definitions. Here we describe PanTHERIA, a species-level data set compiled for analysis of life history, ecology, and geography of all known extant and recently extinct mammals. PanTHERIA is derived from a database capable of holding multiple geo-referenced values for variables within a species containing 100 740 lines of biological data for extant and recently extinct mammalian species, collected over a period of three years by 20 individuals. PanTHERIA also includes spatial databases o...},
author = {Jones, Kate E. and Bielby, Jon and Cardillo, Marcel and Fritz, Susanne A. and O'Dell, Justin and Orme, C. David L. and Safi, Kamran and Sechrest, Wes and Boakes, Elizabeth H. and Carbone, Chris and Connolly, Christina and Cutts, Michael J. and Foster, Janine K. and Grenyer, Richard and Habib, Michael and Plaster, Christopher A. and Price, Samantha A. and Rigby, Elizabeth A. and Rist, Janna and Teacher, Amber and Bininda-Emonds, Olaf R. P. and Gittleman, John L. and Mace, Georgina M. and Purvis, Andy},
doi = {10.1890/08-1494.1},
isbn = {0012-9658},
issn = {0012-9658},
journal = {Ecology},
mendeley-groups = {UCL PhD/Data papers},
pmid = {28441467},
title = {{PanTHERIA: a species-level database of life history, ecology, and geography of extant and recently extinct mammals}},
year = {2009}
}

@article{Myhrvold2015,
abstract = {Studying life-history traits within and across taxonomic classifications has revealed many interesting and important patterns, but this approach to life history requires access to large compilations of data containing many different life-history parameters. Currently, life-history data for amniotes (birds, mammals, and reptiles) are split among a variety of publicly available databases, data tables embedded in individual papers and books, and species-specific studies by experts. Using data from this wide range of sources is a challenge for conducting macroecological studies because of a lack of standardization in taxonomic classifications, parameter values, and even in which parameters are reported. In order to facilitate comparative analyses between amniote life-history data, we created a database compiled from peer-reviewed studies on individual species, macroecological studies of multiple species, existing life-history databases, and other aggregated sources as well as published books and other compila...},
author = {Myhrvold, Nathan P. and Baldridge, Elita and Chan, Benjamin and Sivam, Dhileep and Freeman, Daniel L. and Ernest, S. K. Morgan},
doi = {10.1890/15-0846R.1},
isbn = {0012-9658},
issn = {0012-9658},
journal = {Ecology},
mendeley-groups = {UCL PhD/Data papers},
title = {{An amniote life-history database to perform comparative analyses with birds, mammals, and reptiles}},
year = {2015}
}

@article{Oliveira2017,
abstract = {Current ecological and evolutionary research are increasingly moving from species- to trait-based approaches because traits provide a stronger link to organism's function and fitness. Trait databases covering a large number of species are becoming available, but such data remains scarce for certain groups. Amphibians are among the most diverse vertebrate groups on Earth, and constitute an abundant component of major terrestrial and freshwater ecosystems. They are also facing rapid population declines worldwide, which is likely to affect trait composition in local communities, thereby impacting ecosystem processes and services. In this context, we introduce AmphiBIO, a comprehensive database of natural history traits for amphibians worldwide. The database releases information on 17 traits related to ecology, morphology and reproduction features of amphibians. We compiled data from more than 1,500 literature sources, and for more than 6,500 species of all orders (Anura, Caudata and Gymnophiona), 61 families and 531 genera. This database has the potential to allow unprecedented large-scale analyses in ecology, evolution, and conservation of amphibians.},
author = {Oliveira, Brunno Freire and S{\~{a}}o-Pedro, Vin{\'{i}}cius Avelar and Santos-Barrera, Georgina and Penone, Caterina and Costa, Gabriel C.},
doi = {10.1038/sdata.2017.123},
isbn = {2052-4463 (Electronic) 2052-4463 (Linking)},
issn = {20524463},
journal = {Scientific Data},
mendeley-groups = {UCL PhD/Data papers},
pmid = {28872632},
title = {{AmphiBIO, a global database for amphibian ecological traits}},
year = {2017}
}

@article{Cooke2019a,
abstract = {Species, and their ecological strategies, are disappearing. Here we use species traits to quantify the current and projected future ecological strategy diversity for 15,484 land mammals and birds. We reveal an ecological strategy surface, structured by life-history (fast–slow) and body mass (small–large) as one major axis, and diet (invertivore–herbivore) and habitat breadth (generalist–specialist) as the other. We also find that of all possible trait combinations, only 9% are currently realized. Based on species' extinction probabilities, we predict this limited set of viable strategies will shrink further over the next 100 years, shifting the mammal and bird species pool towards small, fast-lived, highly fecund, insect-eating, generalists. In fact, our results show that this projected decline in ecological strategy diversity is much greater than if species were simply lost at random. Thus, halting the disproportionate loss of ecological strategies associated with highly threatened animals represents a key challenge for conservation.},
author = {Cooke, Robert S.C. and Eigenbrod, Felix and Bates, Amanda E.},
doi = {10.1038/s41467-019-10284-z},
issn = {20411723},
journal = {Nature Communications},
mendeley-groups = {UCL PhD},
title = {{Projected losses of global mammal and bird ecological strategies}},
year = {2019}
}

@article{Cooke2019b,
author = {Cooke, Robert S.C. and Bates, Amanda E. and Eigenbrod, Felix},
doi = {10.1111/geb.12869},
issn = {14668238},
journal = {Global Ecology and Biogeography},
keywords = {birds,ecoregion,function,insurance,mammals,response diversity,traits},
mendeley-groups = {UCL PhD/Papers,UCL PhD},
number = {October 2018},
pages = {1--12},
title = {{Global trade-offs of functional redundancy and functional dispersion for birds and mammals}},
year = {2019}
}

@article{Gonzalez2018,
abstract = {Abstract Aim Collisions between wildlife and vehicles are recognized as one of the major causes of mortality for many species. Empirical estimates of road mortality show that some species are more likely to be killed than others, but to what extent this variation can be explained and predicted using intrinsic species characteristics remains poorly understood. This study aims to identify general macroecological patterns associated with road mortality and generate spatial and species-level predictions of risks. Location Brazil. Time period 2001–2014. Major taxa Birds and mammals. Methods We fitted trait-based random forest regression models (controlling for survey characteristics) to explain 783 empirical road mortality rates from Brazil, representing 170 bird and 73 mammalian species. Fitted models were then used to make spatial and species-level predictions of road mortality risk in Brazil, considering 1,775 birds and 623 mammals that occur within the continental boundaries of the country. Results Survey frequency and geographical location were key predictors of observed rates, but mortality was also explained by the body size, reproductive speed and ecological specialization of the species. Spatial predictions revealed a high potential standardized (per kilometre of road) mortality risk in Amazonia for birds and mammals and, additionally, a high risk in Southern Brazil for mammals. Given the existing road network, these predictions mean that >8 million birds and >2 million mammals could be killed per year on Brazilian roads. Furthermore, predicted rates for all Brazilian endotherms uncovered potential vulnerability to road mortality of several understudied species that are currently listed as threatened by the International Union for Conservation of Nature. Conclusion With a rapidly expanding global road network, there is an urgent need to develop improved approaches to assess and predict road-related impacts. This study illustrates the potential of trait-based models as assessment tools to gain a better understanding of the correlates of vulnerability to road mortality across species, and as predictive tools for difficult-to-sample or understudied species and areas.},
author = {Gonz{\'{a}}lez-Su{\'{a}}rez, Manuela and {Zanchetta Ferreira}, Fl{\'{a}}vio and Grilo, Clara},
doi = {10.1111/geb.12769},
journal = {Global Ecology and Biogeography},
keywords = {Brazil,bird,life history,machine-learning models,mammal,random forest,road-associated mortality,species traits},
number = {9},
pages = {1093--1105},
title = {{Spatial and species-level predictions of road mortality risk using trait data}},
url = {https://onlinelibrary.wiley.com/doi/abs/10.1111/geb.12769},
volume = {27},
year = {2018}
}

@misc{Schneider2019,
abstract = {Trait-based approaches are widespread throughout ecological research as they offer great potential to achieve a general understanding of a wide range of ecological and evolutionary mechanisms. Accordingly, a wealth of trait data is available for many organism groups, but this data is underexploited due to a lack of standardization and heterogeneity in data formats and definitions. We review current initiatives and structures developed for standardizing trait data and discuss the importance of standardization for trait data hosted in distributed open-access repositories. In order to facilitate the standardization and harmonization of distributed trait datasets by data providers and data users, we propose a standardized vocabulary that can be used for storing and sharing ecological trait data. We discuss potential incentives and challenges for the wide adoption of such a standard by data providers. The use of a standard vocabulary allows for trait datasets from heterogeneous sources to be aggregated more easily into compilations and facilitates the creation of interfaces between software tools for trait-data handling and analysis. By aiding decentralized trait-data standardization, our vocabulary may ease data integration and use of trait data for a broader ecological research community and enable global syntheses across a wide range of taxa and ecosystems.},
author = {Schneider, Florian D. and Fichtmueller, David and Gossner, Martin M. and G{\"{u}}ntsch, Anton and Jochum, Malte and K{\"{o}}nig-Ries, Birgitta and {Le Provost}, Ga{\"{e}}tane and Manning, Peter and Ostrowski, Andreas and Penone, Caterina and Simons, Nadja K.},
booktitle = {Methods in Ecology and Evolution},
doi = {10.1111/2041-210X.13288},
issn = {2041210X},
keywords = {data standardization,ecoinformatics,functional ecology,ontologies,semantic web,species traits},
title = {{Towards an ecological trait-data standard}},
year = {2019}
}

@article{Hortal2015,
abstract = {Ecologists and evolutionary biologists are increasingly using big-data approaches to tackle questions at large spatial, taxonomic, and temporal scales. However, despite recent efforts to gather two centuries of biodiversity inventories into comprehensive databases, many crucial research questions remain unanswered. Here, we update the concept of knowledge shortfalls and review the tradeoffs between generality and uncertainty. We present seven key shortfalls of current biodiversity data. Four previously proposed shortfalls pinpoint knowledge gaps for species taxonomy (Linnean), distribution (Wallacean), abundance (Prestonian), and evolutionary patterns (Darwinian). We also redefine the Hutchinsonian shortfall to apply to the abiotic tolerances of species and propose new shortfalls relating to limited knowledge of species traits (Raunkiaeran) and biotic interactions (Eltonian). We conclude with a general framework for the combined impacts and consequences of shortfalls of large-scale biodiversity knowledge ...},
author = {Hortal, Joaqu{\'{i}}n and de Bello, Francesco and Diniz-Filho, Jos{\'{e}} Alexandre Felizola and Lewinsohn, Thomas M. and Lobo, Jorge M. and Ladle, Richard J.},
doi = {10.1146/annurev-ecolsys-112414-054400},
isbn = {1543-592X},
issn = {1543-592X},
journal = {Annual Review of Ecology, Evolution, and Systematics},
mendeley-groups = {UCL PhD/Papers},
title = {{Seven Shortfalls that Beset Large-Scale Knowledge of Biodiversity}},
year = {2015}
}

@article{Gonzalez-Suarez2012,
abstract = {1. Comparative analyses are used to address the key question of what makes a species more prone to extinction by exploring the links between vulnerability and intrinsic species' traits and/or extrinsic factors. This approach requires comprehensive species data but information is rarely available for all species of interest. As a result comparative analyses often rely on subsets of relatively few species that are assumed to be representative samples of the overall studied group. 2. Our study challenges this assumption and quantifies the taxonomic, spatial, and data type biases associated with the quantity of data available for 5415 mammalian species using the freely available life-history database PanTHERIA. 3. Moreover, we explore how existing biases influence results of comparative analyses of extinction risk by using subsets of data that attempt to correct for detected biases. In particular, we focus on links between four species' traits commonly linked to vulnerability (distribution range area, adult body mass, population density and gestation length) and conduct univariate and multivariate analyses to understand how biases affect model predictions. 4. Our results show important biases in data availability with c.22% of mammals completely lacking data. Missing data, which appear to be not missing at random, occur frequently in all traits (14-99% of cases missing). Data availability is explained by intrinsic traits, with larger mammals occupying bigger range areas being the best studied. Importantly, we find that existing biases affect the results of comparative analyses by overestimating the risk of extinction and changing which traits are identified as important predictors. 5. Our results raise concerns over our ability to draw general conclusions regarding what makes a species more prone to extinction. Missing data represent a prevalent problem in comparative analyses, and unfortunately, because data are not missing at random, conventional approaches to fill data gaps, are not valid or present important challenges. These results show the importance of making appropriate inferences from comparative analyses by focusing on the subset of species for which data are available. Ultimately, addressing the data bias problem requires greater investment in data collection and dissemination, as well as the development of methodological approaches to effectively correct existing biases.},
author = {Gonz{\'{a}}lez-Su{\'{a}}rez, Manuela and Lucas, Pablo M. and Revilla, Eloy},
doi = {10.1111/j.1365-2656.2012.01999.x},
issn = {00218790},
journal = {Journal of Animal Ecology},
keywords = {Data imputation,Extinction risk,Life-history traits,Phylogenetic generalized linear models,Phylopars},
mendeley-groups = {UCL PhD/Papers},
title = {{Biases in comparative analyses of extinction risk: Mind the gap}},
year = {2012}
}

@article{Baraldi2010,
abstract = {A great deal of recent methodological research has focused on two modern missing data analysis methods: maximum likelihood and multiple imputation. These approaches are advantageous to traditional techniques (e.g. deletion and mean imputation techniques) because they require less stringent assumptions and mitigate the pitfalls of traditional techniques. This article explains the theoretical underpinnings of missing data analyses, gives an overview of traditional missing data techniques, and provides accessible descriptions of maximum likelihood and multiple imputation. In particular, this article focuses on maximum likelihood estimation and presents two analysis examples from the Longitudinal Study of American Youth data. One of these examples includes a description of the use of auxiliary variables. Finally, the paper illustrates ways that researchers can use intentional, or planned, missing data to enhance their research designs. {\textcopyright} 2009 Society for the Study of School Psychology.},
author = {Baraldi, Amanda N. and Enders, Craig K.},
doi = {10.1016/j.jsp.2009.10.001},
issn = {00224405},
journal = {Journal of School Psychology},
keywords = {Maximum likelihood,Missing data,Multiple imputation,Planned missingness},
pmid = {20006986},
title = {{An introduction to modern missing data analyses}},
year = {2010}
}

@misc{Nakagawa2008,
abstract = {The most common approach to dealing with missing data is to delete cases containing missing observations. However, this approach reduces statistical power and increases estimation bias. A recent study shows how estimates of heritability and selection can be biased when the 'invisible fraction' (missing data due to mortality) is ignored, thus demonstrating the dangers of neglecting missing data in ecology and evolution. We highlight recent advances in the procedures of handling missing data and their relevance and applicability. {\textcopyright} 2008 Elsevier Ltd. All rights reserved.},
author = {Nakagawa, Shinichi and Freckleton, Robert P.},
booktitle = {Trends in Ecology and Evolution},
doi = {10.1016/j.tree.2008.06.014},
issn = {01695347},
mendeley-groups = {UCL PhD/Papers},
title = {{Missing inaction: the dangers of ignoring missing data}},
year = {2008}
}

@article{Martin2012,
abstract = {Although the geographical context of ecological observations shapes ecological theory, the global distribution of ecological studies has never been analyzed. Here, we document the global distribution and context (protected status, biome, anthrome, and net primary productivity) of 2573 terrestrial study sites reported in recent publications (2004–2009) of 10 highly cited ecology journals. We find evidence of several geographical biases, including overrepresentation of protected areas, temperate deciduous woodlands, and wealthy countries. Even within densely settled or agricultural regions, ecologists tend to study “natural” fragments. Such biases in trendsetting journals may limit the scalability of ecological theory and hinder conservation efforts in the 75\% of the terrestrial world where humans live and work.},
author = {Martin, Laura J and Blossey, Bernd and Ellis, Erle},
doi = {10.1890/110154},
journal = {Frontiers in Ecology and the Environment},
number = {4},
pages = {195--201},
title = {{Mapping where ecologists work: biases in the global distribution of terrestrial ecological observations}},
url = {https://esajournals.onlinelibrary.wiley.com/doi/abs/10.1890/110154},
volume = {10},
year = {2012}
}

@techreport{ONU2015,
abstract = {The UNESCO Science Report: towards 2030 provides more country-level information than ever before. The trends and developments in science, technology and innovation policy and governance between 2009 and mid-2015 described here provide essential baseline information on the concerns and priorities of countries that should orient the implementation and drive the assessment of the 2030 Agenda for Sustainable Development in the years to come. The report includes the chapter "Is the gender gap narrowing in science and engineering?" written by Sophia Huyer, Executive Director of WISAT, which highlights the finding that women are entering agricultural sciences in increasing numbers in almost all regions of the world.},
author = {{United Nations Educational Scientific and Cultural} and Organization},
booktitle = {UNESCO Global Science Report: Towards 2030},
isbn = {9789231001291},
title = {{UNESCO Global Science Report: Towards 2030}},
year = {2015}
}

@article{Collen2008,
abstract = {Nations around the world are required to measure their progress towards key biodiversity goals. One important example of this, the Convention on Biological Diversity's 2010 target, is soon approaching. The target set is to significantly reduce the rate of biodiversity loss by the year 2010. However, to what extent are the data, especially for tropical countries, available to indicate biodiversity change and to what extent is current knowledge of biodiversity change truly a global picture? While species richness is greatest in the tropics, biodiversity data richness is skewed towards the poles. This not only provides a significant challenge for global indicators to accurately represent biodiversity, but also for individual countries that are responsible under such legislation for measuring their own impact on biodiversity. We examine the coverage of biodiversity data using four global biodiversity datasets, and look at how effective current efforts are at addressing this discrepancy, and what countries might be able to do in time for 2010 and beyond. We conclude by suggesting a number of activities which might provide impetus for improved biodiversity monitoring in tropical nations.},
author = {Collen, Ben and Ram, Mala and Zamin, Tara and McRae, Louise},
doi = {10.1177/194008290800100202},
file = {:C\:/Users/adrie/Downloads/194008290800100202.pdf:pdf},
issn = {1940-0829},
journal = {Tropical Conservation Science},
keywords = {biodiversity monitoring,convention on biological diversity,indicators for 2010 target,population decline,threatened species},
number = {2},
pages = {75--88},
title = {{The Tropical Biodiversity Data Gap: Addressing Disparity in Global Monitoring}},
volume = {1},
year = {2008}
}


@Manual{rredlist,
    title = {rredlist: 'IUCN' Red List Client},
    author = {Scott Chamberlain},
    year = {2018},
    note = {R package version 0.5.0},
    url = {https://CRAN.R-project.org/package=rredlist},
  }

@article{IUCN, 
author="IUCN",
title={{The IUCN Red List of Threatened Species.}},
year={2013}, 
volume={Version 2013.7.}, 
doi={http://www.iucnredlist.org.}
}

@article{IUCN2020, 
author="IUCN",
title={{The IUCN Red List of Threatened Species. http://www.iucnredlist.org.}},
year={2020}, 
volume={Version 2020.1.}, 
doi={}
}


@Manual{ESRI,
   title = {ArcGIS desktop: Release 10.},
   author = {{Environmental Systems Research Institute.}},
   year = {2011}
  }

@Manual{R_citation,
   title = {R: A Language and Environment for Statistical Computing},
   author = {{R Core Team}},
   organization = {R Foundation for Statistical Computing},
   address = {Vienna, Austria},
   year = {2018},
   url = {https://www.R-project.org/}
  }
  
  @Manual{Python_citation,
   title = {Python tutorial},
   author = {{van Rossum, G}},
   organization = {Centrum voor Wiskunde en Informatica (CWI)},
   year = {1995}
  }


  @article{Chamberlain2013,
abstract = {All species are hierarchically related to one another, and we use taxonomic names to label the nodes in this hierarchy. Taxonomic data is becoming increasingly available on the web, but scientists need a way to access it in a programmatic fashion that's easy and reproducible. We have developed taxize, an open-source software package (freely available from) for the R language. http://cran.r-project.org/web/packages/taxize/index.html taxize provides simple, programmatic access to taxonomic data for 13 data sources around the web. We discuss the need for a taxonomic toolbelt in R, and outline a suite of use cases for which taxize is ideally suited (including a full workflow as an appendix). The taxize package facilitates open and reproducible science by allowing taxonomic data collection to be done in the open-source R platform.},
author = {Chamberlain, Scott A and Sz{\"{o}}cs, Eduard},
doi = {10.12688/f1000research.2-191.v2},
journal = {F1000Research},
title = {{taxize : taxonomic search and retrieval in R [version 2; referees: 3 approved]}},
year = {2013}
}



@article{Cooper2008,
abstract = {},
author = {Cooper, Natalie and Bielby, Jon and Thomas, Gavin H. and Purvis, Andy},
doi = {10.1111/j.1466-8238.2007.00355.x},
isbn = {1466-8238},
issn = {1466822X},
journal = {Global Ecology and Biogeography},
keywords = {Amphibian,Body size,Clutch size,Conservation,Extinction risk,Geographical range size,Independent contrasts,Spatial autocorrelation},
pmid = {1449},
title = {{Macroecology and extinction risk correlates of frogs}},
year = {2008}
}

@article{Sodhi2008,
abstract = {},
author = {Sodhi, Navjot S. and Bickford, David and Diesmos, Arvin C. and Lee, Tien Ming and Koh, Lian Pin and Brook, Barry W. and Sekercioglu, Cagan H. and Bradshaw, Corey J.A.},
doi = {10.1371/journal.pone.0001636},
isbn = {1932-6203},
issn = {19326203},
journal = {PLoS ONE},
pmid = {18286193},
title = {{Measuring the meltdown: Drivers of global amphibian extinction and decline}},
year = {2008}
}

@article{Wilman2014,
abstract = {},
author = {Wilman, Hamish and Belmaker, Jonathan and Simpson, Jennifer and de la Rosa, Carolina and Rivadeneira, Marcelo M. and Jetz, Walter},
doi = {10.1890/13-1917.1},
isbn = {0012-9658},
issn = {0012-9658},
journal = {Ecology},
pmid = {17435515},
title = {{EltonTraits 1.0: Species-level foraging attributes of the world's birds and mammals}},
year = {2014}
}


@article{BirdLife, 
author="BirdLife International, NatureServe",
title={{Bird species distribution maps of the world. Version 2.0.}},
year={2012}, 
doi={See:http://www.birdlife.org/datazone/info/spcdownload.}
}

@article{Kissling2014,
abstract = {},
author = {Kissling, Wilm Daniel and Dalby, Lars and Fl{\o}jgaard, Camilla and Lenoir, Jonathan and Sandel, Brody and Sandom, Christopher and Tr{\o}jelsgaard, Kristian and Svenning, Jens Christian},
doi = {10.1002/ece3.1136},
isbn = {2045-7758},
issn = {20457758},
journal = {Ecology and Evolution},
keywords = {Diet ecology,Ecological trait data,Feeding guild,Mammalia,Phylogenetic conservatism,Trophic structure},
pmid = {25165528},
title = {{Establishing macroecological trait datasets: Digitalization, extrapolation, and validation of diet preferences in terrestrial mammals worldwide}},
year = {2014}
}

@article{Gainsbury2018,
abstract = {},
author = {Gainsbury, Alison M. and Tallowin, Oliver J.S. and Meiri, Shai},
journal = {Mammal Review},
doi = {10.1111/mam.12119},
issn = {13652907},
keywords = {dietary data set,dietary guilds,extrapolation validity,mammalia,trophic level},
title = {{An updated global data set for diet preferences in terrestrial mammals: testing the validity of extrapolation}},
year = {2018}
}
@article{Scharf2015,
abstract = {},
author = {Scharf, Inon and Feldman, Anat and Novosolov, Maria and Pincheira-Donoso, Daniel and Das, Indraneil and B{\"{o}}hm, Monika and Uetz, Peter and Torres-Carvajal, Omar and Bauer, Aaron and Roll, Uri and Meiri, Shai},
doi = {10.1111/geb.12244},
issn = {14668238},
journal = {Global Ecology and Biogeography},
keywords = {Body size,Fast-slow continuum,Lifespan,NPP,Phylogenetic comparisons,Reproduction,Reptiles,Temperature,Trade-off},
title = {{Late bloomers and baby boomers: Ecological drivers of longevity in squamates and the tuatara}},
year = {2015}
}

@article{Vidan2017,
abstract = {},
author = {Vidan, Enav and Roll, Uri and Bauer, Aaron and Grismer, Lee and Guo, Peng and Maza, Erez and Novosolov, Maria and Sindaco, Roberto and Wagner, Philipp and Belmaker, Jonathan and Meiri, Shai},
doi = {10.1111/geb.12643},
issn = {14668238},
journal = {Global Ecology and Biogeography},
keywords = {ambient temperature hypothesis,night temperature,productivity hypothesis,richness},
title = {{The Eurasian hot nightlife: Environmental forces associated with nocturnality in lizards}},
year = {2017}
}

@article{Stark2018,
abstract = {},
author = {Stark, Gavin and Tamar, Karin and Itescu, Yuval and Feldman, Anat and Meiri, Shai},
doi = {10.1093/biolinnean/bly153/5145102},
isbn = {153/5145102},
issn = {1461-6742},
journal = {Biological Journal of the Linnean Society},
title = {{Cold and isolated ectotherms: drivers of reptilian longevity}},
year = {2018}
}

@article{Schwarz2017,
abstract = {},
author = {Schwarz, Rachel and Meiri, Shai},
doi = {10.1111/jbi.13067},
issn = {13652699},
journal = {Journal of Biogeography},
keywords = {anoles,egg volume,geckos,invariant clutch size,island biogeography,island syndrome,life-history,lizards,reproduction,reversed island syndrome},
title = {{The fast-slow life-history continuum in insular lizards: a comparison between species with invariant and variable clutch sizes}},
year = {2017}
}

@article{Novosolov2013,
abstract = {},
author = {Novosolov, Maria and Raia, Pasquale and Meiri, Shai},
doi = {10.1111/j.1466-8238.2012.00791.x},
isbn = {1466-8238},
issn = {1466822X},
journal = {Global Ecology and Biogeography},
keywords = {Clutch size,Island biogeography,Island syndrome,Life history,Lizards,Population density,Reproduction,Reversed island syndrome},
pmid = {19944116},
title = {{The island syndrome in lizards}},
year = {2013}
}

@article{Novosolov2017,
abstract = {},
author = {Novosolov, Maria and Rodda, Gordon H. and North, Alexandra C. and Butchart, Stuart H.M. and Tallowin, Oliver J.S. and Gainsbury, Alison M. and Meiri, Shai},
doi = {10.1111/geb.12617},
isbn = {1466822X},
issn = {14668238},
journal = {Global Ecology and Biogeography},
keywords = {birds,ecological rule,lizards,mammals,population density,range size},
title = {{Population density–range size relationship revisited}},
year = {2017}
}

@article{Slavenko2016,
abstract = {},
author = {Slavenko, Alex and Tallowin, Oliver J.S. and Itescu, Yuval and Raia, Pasquale and Meiri, Shai},
doi = {10.1111/geb.12491},
isbn = {1466-8238},
issn = {14668238},
journal = {Global Ecology and Biogeography},
keywords = {Body size,Holocene extinction,Quaternary,conservation,global,megafaunal extinctions,reptiles},
title = {{Late Quaternary reptile extinctions: size matters, insularity dominates}},
year = {2016}
}

@article{FeldmanGEB2016,
author = {Feldman, Anat and Sabath, Niv and Pyron, R. Alexander and Mayrose, Itay and Meiri, Shai},
title = {Body sizes and diversification rates of lizards, snakes, amphisbaenians and the tuatara},
journal = {Global Ecology and Biogeography},
volume = {25},
number = {2},
pages = {187-197},
keywords = {Body size–frequency distributions, diversification rates, extinction rates, lizards, mass, snakes, speciation rates},
doi = {10.1111/geb.12398},
url = {https://onlinelibrary.wiley.com/doi/abs/10.1111/geb.12398},
eprint = {https://onlinelibrary.wiley.com/doi/pdf/10.1111/geb.12398},
abstract = {},
year = {2016}
}


@article{Meiri2018GEB,
author = {Meiri, Shai},
title = {Traits of lizards of the world: Variation around a successful evolutionary design},
journal = {Global Ecology and Biogeography},
volume = {27},
number = {10},
pages = {1168-1172},
keywords = {distribution, ecology, life history, morphology, natural history, Sauria, thermal biology, traits},
doi = {10.1111/geb.12773},
url = {https://onlinelibrary.wiley.com/doi/abs/10.1111/geb.12773},
eprint = {https://onlinelibrary.wiley.com/doi/pdf/10.1111/geb.12773},
abstract = {},
year = {2018}
}
  
@article{Meiri2015,
author = {Meiri, S. and Feldman, A. and Kratochvíl, L.},
title = {Squamate hatchling size and the evolutionary causes of negative offspring size allometry},
journal = {Journal of Evolutionary Biology},
volume = {28},
number = {2},
pages = {438-446},
keywords = {Anolis, body size, clutch size, egg size, geckos, life history, relative clutch mass, reproduction, Squamata},
doi = {10.1111/jeb.12580},
url = {https://onlinelibrary.wiley.com/doi/abs/10.1111/jeb.12580},
eprint = {https://onlinelibrary.wiley.com/doi/pdf/10.1111/jeb.12580},
abstract = {},
year = {2015}
}

@article{Roll2017,
abstract = {},
author = {Roll, Uri and Feldman, Anat and Novosolov, Maria and Allison, Allen and Bauer, Aaron M. and Bernard, Rodolphe and B{\"{o}}hm, Monika and Castro-Herrera, Fernando and Chirio, Laurent and Collen, Ben and Colli, Guarino R. and Dabool, Lital and Das, Indraneil and Doan, Tiffany M. and Grismer, Lee L. and Hoogmoed, Marinus and Itescu, Yuval and Kraus, Fred and Lebreton, Matthew and Lewin, Amir and Martins, Marcio and Maza, Erez and Meirte, Danny and Nagy, Zolt{\'{a}}n T. and Nogueira, Cristiano De C. and Pauwels, Olivier S.G. and Pincheira-Donoso, Daniel and Powney, Gary D. and Sindaco, Roberto and Tallowin, Oliver J.S. and Torres-Carvajal, Omar and Trape, Jean Fran{\c{c}}ois and Vidan, Enav and Uetz, Peter and Wagner, Philipp and Wang, Yuezhao and Orme, C. David L. and Grenyer, Richard and Meiri, Shai},
doi = {10.1038/s41559-017-0332-2},
issn = {2397334X},
journal = {Nature Ecology and Evolution},
pmid = {28993667},
title = {{The global distribution of tetrapods reveals a need for targeted reptile conservation}},
year = {2017}
}


  @article{Gnanadesikan2017,
author = {Gnanadesikan, Gitanjali E. and Pearse, William D. and Shaw, Allison K.},
title = {Evolution of mammalian migrations for refuge, breeding, and food},
journal = {Ecology and Evolution},
volume = {7},
number = {15},
pages = {5891--5900},
keywords = {body mass, conservation, diet, IUCN Red List, movement ecology, seasonal migration, tracking},
doi = {10.1002/ece3.3120},
abstract = {},
year = {2017}
}










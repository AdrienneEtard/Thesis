%% Bibliography for SI

@article{Etard2020,
abstract = {Aim: Trait data are increasingly being used in studies investigating the impacts of global changes on the structure and functioning of ecological communities. Despite a growing number of trait data collations for terrestrial vertebrates, there is to date no global assessment of the gaps and biases the data present. Here, we assess whether terrestrial vertebrate trait data are taxonomically, spatially and phylogenetically biased. Location: Global. Time period: Present. Major taxa studied: Terrestrial vertebrates. Methods: We compile seven ecological traits and quantify coverage as the proportion of species for which an estimate is available. For a species, we define completeness as the proportion of non-missing values across traits. We assess whether coverage and completeness differ across classes and examine phylogenetic biases in trait data. To investigate spatial biases, we test whether wider-ranging species have more complete trait data than narrow-ranging species. Additionally, we test whether species-rich regions, which are of most concern for conservation, are less well sampled than species-poor regions. Results: Mammals and birds are well sampled even in species-rich regions. For reptiles and amphibians (herptiles), only body size presents a high coverage (>80%), in addition to habitat-related variables (amphibians). Herptiles are poorly sampled for other traits. The shortfalls are particularly acute in some species-rich regions and for certain clades. Across all classes, geographically rarer species have less complete trait information. Main conclusions: Trait information is less available on average in some of the most diverse areas and in geographically rarer species, both of which crucial for biodiversity conservation. Gaps in trait data might impede our ability to conduct large-scale analyses, whereas biases can impact the validity of extrapolations. A short-term solution to the problem is to estimate missing trait data using imputation techniques, whereas a longer-term and more robust filling of existing gaps requires continued data-collection efforts.},
author = {Etard, Adrienne and Morrill, Sophie and Newbold, Tim},
doi = {10.1111/geb.13184},
file = {:C\:/Users/adrie/AppData/Local/Mendeley Ltd./Mendeley Desktop/Downloaded/Etard, Morrill, Newbold - 2020 - Global gaps in trait data for terrestrial vertebrates.pdf:pdf},
issn = {14668238},
journal = {Global Ecology and Biogeography},
keywords = {completeness,coverage,phylogenetic biases,spatial biases,taxonomic biases,terrestrial vertebrates,traits},
number = {November 2019},
pages = {1--16},
title = {{Global gaps in trait data for terrestrial vertebrates}},
year = {2020}
}

   @article{mice,
    title = {{mice}: Multivariate Imputation by Chained Equations in R},
    author = {Stef {van Buuren} and Karin Groothuis-Oudshoorn},
    journal = {Journal of Statistical Software},
    year = {2011},
    volume = {45},
    number = {3},
    pages = {1-67},
    url = {https://www.jstatsoft.org/v45/i03/},
  }
  
  @article{Troyanskaya2001,
abstract = {MOTIVATION: Gene expression microarray experiments can generate data sets with multiple missing expression values. Unfortunately, many algorithms for gene expression analysis require a complete matrix of gene array values as input. For example, methods such as hierarchical clustering and K-means clustering are not robust to missing data, and may lose effectiveness even with a few missing values. Methods for imputing missing data are needed, therefore, to minimize the effect of incomplete data sets on analyses, and to increase the range of data sets to which these algorithms can be applied. In this report, we investigate automated methods for estimating missing data. RESULTS: We present a comparative study of several methods for the estimation of missing values in gene microarray data. We implemented and evaluated three methods: a Singular Value Decomposition (SVD) based method (SVDimpute), weighted K-nearest neighbors (KNNimpute), and row average. We evaluated the methods using a variety of parameter settings and over different real data sets, and assessed the robustness of the imputation methods to the amount of missing data over the range of 1--20% missing values. We show that KNNimpute appears to provide a more robust and sensitive method for missing value estimation than SVDimpute, and both SVDimpute and KNNimpute surpass the commonly used row average method (as well as filling missing values with zeros). We report results of the comparative experiments and provide recommendations and tools for accurate estimation of missing microarray data under a variety of conditions.},
author = {Troyanskaya, Olga and Cantor, Michael and Sherlock, Gavin and Brown, Pat and Hastie, Trevor and Tibshirani, Robert and Botstein, David and Altman, Russ B.},
doi = {10.1093/bioinformatics/17.6.520},
isbn = {1367-4803 (Print)\r1367-4803},
issn = {13674803},
journal = {Bioinformatics},
pmid = {11395428},
title = {{Missing value estimation methods for DNA microarrays}},
year = {2001}
}

@article{Stekhoven2012,
abstract = {Modern data acquisition based on high-throughput technology is often facing the problem of missing data. Algorithms commonly used in the analysis of such large-scale data often depend on a complete set. Missing value imputation offers a solution to this problem. However, the majority of available imputation methods are restricted to one type of variable only: continuous or categorical. For mixed-type data the different types are usually handled separately. Therefore, these methods ignore possible relations between variable types. We propose a nonparametric method which can cope with different types of variables simultaneously. We compare several state of the art methods for the imputation of missing values. We propose and evaluate an iterative imputation method (missForest) based on a random forest. By averaging over many unpruned classification or regression trees random forest intrinsically constitutes a multiple imputation scheme. Using the built-in out-of-bag error estimates of random forest we are able to estimate the imputation error without the need of a test set. Evaluation is performed on multiple data sets coming from a diverse selection of biological fields with artificially introduced missing values ranging from 10% to 30%. We show that missForest can successfully handle missing values, particularly in data sets including different types of variables. In our comparative study missForest outperforms other methods of imputation especially in data settings where complex interactions and nonlinear relations are suspected. The out-of-bag imputation error estimates of missForest prove to be adequate in all settings. Additionally, missForest exhibits attractive computational efficiency and can cope with high-dimensional data.},
archivePrefix = {arXiv},
arxivId = {1105.0828},
author = {Stekhoven, Daniel J. and B{\"{u}}hlmann, Peter},
doi = {10.1093/bioinformatics/btr597},
eprint = {1105.0828},
isbn = {1367-4811 (Electronic)\n1367-4803 (Linking)},
issn = {13674803},
journal = {Bioinformatics},
pmid = {22039212},
title = {{Missforest-Non-parametric missing value imputation for mixed-type data}},
year = {2012}
}

@article{Stekhoven2016,
abstract = {MOTIVATION: Modern data acquisition based on high-throughput technology is often facing the problem of missing data. Algorithms commonly used in the analysis of such large-scale data often depend on a complete set. Missing value imputation offers a solution to this problem. However, the majority of available imputation methods are restricted to one type of variable only: continuous or categorical. For mixed-type data, the different types are usually handled separately. Therefore, these methods ignore possible relations between variable types. We propose a non-parametric method which can cope with different types of variables simultaneously. RESULTS: We compare several state of the art methods for the imputation of missing values. We propose and evaluate an iterative imputation method (missForest) based on a random forest. By averaging over many unpruned classification or regression trees, random forest intrinsically constitutes a multiple imputation scheme. Using the built-in out-of-bag error estimates of random forest, we are able to estimate the imputation error without the need of a test set. Evaluation is performed on multiple datasets coming from a diverse selection of biological fields with artificially introduced missing values ranging from 10% to 30%. We show that missForest can successfully handle missing values, particularly in datasets including different types of variables. In our comparative study, missForest outperforms other methods of imputation especially in data settings where complex interactions and non-linear relations are suspected. The out-of-bag imputation error estimates of missForest prove to be adequate in all settings. Additionally, missForest exhibits attractive computational efficiency and can cope with high-dimensional data. AVAILABILITY: The package missForest is freely available from http://stat.ethz.ch/CRAN/. CONTACT: stekhoven@stat.math.ethz.ch; buhlmann@stat.math.ethz.ch},
archivePrefix = {arXiv},
arxivId = {1105.0828},
author = {Stekhoven, Daniel J.},
doi = {10.1093/bioinformatics/btr597},
eprint = {1105.0828},
isbn = {1367-4811 (Electronic)\n1367-4803 (Linking)},
issn = {13674803},
journal = {R Package version 1.4},
pmid = {22039212},
title = {{Nonparametric Missing Value Imputation using Random Forest}},
year = {2016}
}

@article{Bruggeman2009,
abstract = {A wealth of information on metabolic parameters of a species can be inferred from observations on species that are phylogenetically related. Phylogeny-based information can complement direct empirical evidence, and is particularly valuable if experiments on the species of interest are not feasible. The PhyloPars web server provides a statistically consistent method that combines an incomplete set of empirical observations with the species phylogeny to produce a complete set of parameter estimates for all species. It builds upon a state-of-the-art evolutionary model, extended with the ability to handle missing data. The resulting approach makes optimal use of all available information to produce estimates that can be an order of magnitude more accurate than ad-hoc alternatives. Uploading a phylogeny and incomplete feature matrix suffices to obtain estimates of all missing values, along with a measure of certainty. Real-time cross-validation provides further insight in the accuracy and bias expected for estimated values. The server allows for easy, efficient estimation of metabolic parameters, which can benefit a wide range of fields including systems biology and ecology. PhyloPars is available at: http://www.ibi.vu.nl/programs/phylopars/.},
author = {Bruggeman, Jorn and Heringa, Jaap and Brandt, Bernd W.},
doi = {10.1093/nar/gkp370},
isbn = {03051048 (ISSN)},
issn = {03051048},
journal = {Nucleic Acids Research},
pmid = {19443453},
title = {{PhyloPars: Estimation of missing parameter values using phylogeny}},
year = {2009}
}

@article{Penone2014,
abstract = {* Despite efforts in data collection, missing values are commonplace in life-history trait databases. Because these values typically are not missing randomly, the common practice of removing missing data not only reduces sample size, but also introduces bias that can lead to incorrect conclusions. Imputing missing values is a potential solution to this problem. Here, we evaluate the performance of four approaches for estimating missing values in trait databases (K-nearest neighbour (kNN), multivariate imputation by chained equations (mice), missForest and Phylopars), and test whether imputed datasets retain underlying allometric relationships among traits.\n\n* Starting with a nearly complete trait dataset on the mammalian order Carnivora (using four traits), we artificially removed values so that the percent of missing values ranged from 10% to 80%. Using the original values as a reference, we assessed imputation performance using normalized root mean squared error. We also evaluated whether including phylogenetic information improved imputation performance in kNN, mice, and missForest (it is a required input in Phylopars). Finally, we evaluated the extent to which the allometric relationship between two traits (body mass and longevity) was conserved for imputed datasets by looking at the difference (bias) between the slope of the original and the imputed datasets or datasets with missing values removed.\n\n* Three of the tested approaches (mice, missForest and Phylopars), resulted in qualitatively equivalent imputation performance, and all had significantly lower errors than kNN. Adding phylogenetic information into the imputation algorithms improved estimation of missing values for all tested traits. The allometric relationship between body mass and longevity was conserved when up to 60% of data were missing, either with or without phylogenetic information, depending on the approach. This relationship was less biased in imputed datasets compared to datasets with missing values removed, especially when more than 30% of values were missing.\n\n* Imputations provide valuable alternatives to removing missing observations in trait databases as they produce low errors and retain relationships among traits. Although we must continue to prioritize data collection on species traits, imputations can provide a valuable solution for conducting macroecological and evolutionary studies using life-history trait databases.},
author = {Penone, Caterina and Davidson, Ana D. and Shoemaker, Kevin T. and {Di Marco}, Moreno and Rondinini, Carlo and Brooks, Thomas M. and Young, Bruce E. and Graham, Catherine H. and Costa, Gabriel C.},
doi = {10.1111/2041-210X.12232},
isbn = {2041-210x},
issn = {2041210X},
journal = {Methods in Ecology and Evolution},
keywords = {Body mass,Carnivores,KNN,Longevity,MissForest,Multivariate imputation by chained equations,Phylogeny,Phylopars,Rootmean squared error},
pmid = {18823677},
title = {{Imputation of missing data in life-history trait datasets: Which approach performs the best?}},
year = {2014}
}
@article{Pagel1999,
abstract = {Phylogenetic trees describe the pattern of descent amongst a group of species. With the rapid accumulation of DNA sequence data, more and more phylogenies are being constructed based upon sequence comparisons. The combination of these phylogenies with powerful new statistical approaches for the analysis of biological evolution is challenging widely held beliefs about the history and evolution of life on Earth.},
author = {Pagel, Mark},
journal = {Nature},
doi = {10.1038/44766},
isbn = {0028-0836},
issn = {00280836},
pmid = {10553904},
title = {{Inferring the historical patterns of biological evolution}},
year = {1999}
}

@article{Borges2018,
abstract = {MotivationDetermining whether a trait and phylogeny share some degree of phylogenetic signal is a flagship goal in evolutionary biology. Signatures of phylogenetic signal can assist the resolution of a broad range of evolutionary questions regarding the tempo and mode of phenotypic evolution. However, despite the considerable number of strategies to measure it, few and limited approaches exist for categorical traits. Here, we used the concept of Shannon entropy and propose the $\delta$ statistic for evaluating the degree of phylogenetic signal between a phylogeny and categorical traits.ResultsWe validated $\delta$ as a measure of phylogenetic signal: the higher the $\delta$-value the higher the degree of phylogenetic signal between a given tree and a trait. Based on simulated data we proposed a threshold-based classification test to pinpoint cases of phylogenetic signal. The assessment of the test's specificity and sensitivity suggested that the $\delta$ approach should only be applied to 20 or more species. We have further tested the performance of $\delta$ in scenarios of branch length and topology uncertainty, unbiased and biased trait evolution and trait saturation. Our results showed that $\delta$ may be applied in a wide range of phylogenetic contexts. Finally, we investigated our method in 14 360 mammalian gene trees and found that olfactory receptor genes are significantly associated with the mammalian activity patterns, a result that is congruent with expectations and experiments from the literature. Our application shows that $\delta$ can successfully detect molecular signatures of phenotypic evolution. We conclude that $\delta$ represents a useful measure of the strength of phylogenetic signal since many phenotypes can only be measured in categories.Availabilityhttps://github.com/mrborges23/delta_statisticSupplementary informationSupplementary data are available at Bioinformatics online.},
author = {Borges, Rui and Machado, Jo{\~{a}}o Paulo and Gomes, Cid{\'{a}}lia and Rocha, Ana Paula and Antunes, Agostinho},
doi = {10.1093/bioinformatics/bty800},
issn = {1367-4803},
journal = {Bioinformatics},
title = {{Measuring phylogenetic signal between categorical traits and phylogenies}},
year = {2018}
}

@article{Revell2016,
abstract = {},
author = {Revell, Liam J},
journal = {R topics documented},
title = {{Package ‘phytools'}},
year = {2016}
}

@article{Faurby2018,
abstract = {Abstract Data needed for macroecological analyses are difficult to compile and often hidden away in supplementary material under non-standardized formats. Phylogenies, range data, and trait data often use conflicting taxonomies and require ad hoc decisions to synonymize species or fill in large amounts of missing data. Furthermore, most available data sets ignore the large impact that humans have had on species ranges and diversity. Ignoring these impacts can lead to drastic differences in diversity patterns and estimates of the strength of biological rules. To help overcome these issues, we assembled PHYLACINE, The Phylogenetic Atlas of Mammal Macroecology. This taxonomically integrated platform contains phylogenies, range maps, trait data, and threat status for all 5,831 known mammal species that lived since the last interglacial ($\sim$130,000 years ago until present). PHYLACINE is ready to use directly, as all taxonomy and metadata are consistent across the different types of data, and files are provided in easy-to-use formats. The atlas includes both maps of current species ranges and present natural ranges, which represent estimates of where species would live without anthropogenic pressures. Trait data include body mass and coarse measures of life habit and diet. Data gaps have been minimized through extensive literature searches and clearly labelled imputation of missing values. The PHYLACINE database will be archived here as well as hosted online so that users may easily contribute updates and corrections to continually improve the data. This database will be useful to any researcher who wishes to investigate large-scale ecological patterns. Previous versions of the database have already provided valuable information and have, for instance, shown that megafauna extinctions caused substantial changes in vegetation structure and nutrient transfer patterns across the globe.},
author = {Faurby, S{\o}ren and Davis, Matt and Pedersen, Rasmus {\O} and Schowanek, Simon D and Antonelli1, Alexandre and Svenning, Jens-Christian},
doi = {10.1002/ecy.2443},
journal = {Ecology},
keywords = {IUCN,body size,diet,distributions,mammal,mass,phylogeny,present natural,range maps},
number = {11},
pages = {2626},
title = {{PHYLACINE 1.2: The Phylogenetic Atlas of Mammal Macroecology}},
url = {https://esajournals.onlinelibrary.wiley.com/doi/abs/10.1002/ecy.2443},
volume = {99},
year = {2018}
}

@software{Faurby2020,
author = {Faurby, S{\o}ren and Pedersen, Rasmus {\O} and Davis, Matt and Schowanek, Simon D and Jarvie, Scott and Antonelli, Alexandre and Svenning, Jens-Christian},
doi = {10.5281/zenodo.3690867},
month = {feb},
publisher = {Zenodo},
title = {{MegaPast2Future/PHYLACINE\_1.2: PHYLACINE Version 1.2.1}},
url = {https://doi.org/10.5281/zenodo.3690867},
year = {2020}
}

@article{Jetz2012,
abstract = {Current global patterns of biodiversity result from processes that operate over both space and time and thus require an integrated macroecological and macroevolutionary perspective. Molecular time trees have advanced our understanding of the tempo and mode of diversification and have identified remarkable adaptive radiations across the tree of life. However, incomplete joint phylogenetic and geographic sampling has limited broad-scale inference. Thus, the relative prevalence of rapid radiations and the importance of their geographic settings in shaping global biodiversity patterns remain unclear. Here we present, analyse and map the first complete dated phylogeny of all 9,993 extant species of birds, a widely studied group showing many unique adaptations. We find that birds have undergone a strong increase in diversification rate from about 50 million years ago to the near present. This acceleration is due to a number of significant rate increases, both within songbirds and within other young and mostly temperate radiations including the waterfowl, gulls and woodpeckers. Importantly, species characterized with very high past diversification rates are interspersed throughout the avian tree and across geographic space. Geographically, the major differences in diversification rates are hemispheric rather than latitudinal, with bird assemblages in Asia, North America and southern South America containing a disproportionate number of species from recent rapid radiations. The contribution of rapidly radiating lineages to both temporal diversification dynamics and spatial distributions of species diversity illustrates the benefits of an inclusive geographical and taxonomical perspective. Overall, whereas constituent clades may exhibit slowdowns, the adaptive zone into which modern birds have diversified since the Cretaceous may still offer opportunities for diversification. {\textcopyright} 2012 Macmillan Publishers Limited. All rights reserved.},
author = {Jetz, W. and Thomas, G. H. and Joy, J. B. and Hartmann, K. and Mooers, A. O.},
doi = {10.1038/nature11631},
issn = {00280836},
journal = {Nature},
pmid = {23123857},
title = {{The global diversity of birds in space and time}},
year = {2012}
}
@article{Tonini2016,
abstract = {Macroevolutionary rates of diversification and anthropogenic extinction risk differ vastly throughout the Tree of Life. This results in a highly heterogeneous distribution of Evolutionary distinctiveness (ED) and threat status among species. We examine the phylogenetic distribution of ED and threat status for squamates (amphisbaenians, lizards, and snakes) using fully-sampled phylogenies containing 9574 species and expert-based estimates of threat status for $\sim$4000 species. We ask whether threatened species are more closely related than would be expected by chance and whether high-risk species represent a disproportionate amount of total evolutionary history. We found currently-assessed threat status to be phylogenetically clustered at broad level in Squamata, suggesting it is critical to assess extinction risks for close relatives of threatened lineages. Our findings show no association between threat status and ED, suggesting that future extinctions may not result in a disproportionate loss of evolutionary history. Lizards in degraded tropical regions (e.g., Madagascar, India, Australia, and the West Indies) seem to be at particular risk. A low number of threatened high-ED species in areas like the Amazon, Borneo, and Papua New Guinea may be due to a dearth of adequate risk assessments. It seems we have not yet reached a tipping point of extinction risk affecting a majority of species; 63% of the assessed species are not threatened and 56% are Least Concern. Nonetheless, our results show that immediate efforts should focus on geckos, iguanas, and chameleons, representing 67% of high-ED threatened species and 57% of Unassessed high-ED lineages.},
annote = {Advancing reptile conservation: Addressing knowledge gaps and mitigating key drivers of extinction risk},
author = {Tonini, Jo{\~{a}}o Filipe Riva and Beard, Karen H and Ferreira, Rodrigo Barbosa and Jetz, Walter and Pyron, R Alexander},
doi = {https://doi.org/10.1016/j.biocon.2016.03.039},
issn = {0006-3207},
journal = {Biological Conservation},
keywords = {Conservation,Evolutionary distinctiveness,IUCN Red List,Measures of biodiversity,PASTIS},
pages = {23--31},
title = {{Fully-sampled phylogenies of squamates reveal evolutionary patterns in threat status}},
url = {http://www.sciencedirect.com/science/article/pii/S0006320716301483},
volume = {204},
year = {2016}
}

@article{Jetz2018,
abstract = {Human activities continue to Erode the tree of life, requiring us to prioritize research and conservation. Amphibians represent key victims and bellwethers of global change, and the need for action to conserve them is drastically outpacing knowledge. We provide a phylogeny incorporating nearly all extant amphibians (7,238 species). Current amphibian diversity is composed of both older, depauperate lineages and extensive, more recent tropical radiations found in select clades. Frog and salamander diversification increased strongly after the Cretaceous-Palaeogene boundary, preceded by a potential mass-extinction event in salamanders. Diversification rates of subterranean caecilians varied little over time. Biogeographically, the Afro- and Neotropics harbour a particularly high proportion of Gondwanan relicts, comprising species with high evolutionary distinctiveness (ED). These high-ED species represent a large portion of the branches in the present tree: around 28% of all phylogenetic diversity comes from species in the top 10% of ED. The association between ED and imperilment is weak, but many species with high ED are now imperilled or lack formal threat status, suggesting opportunities for integrating evolutionary position and phylogenetic heritage in addressing the current extinction crisis. By providing a phylogenetic estimate for extant amphibians and identifying their threats and ED, we offer a preliminary basis for a quantitatively informed global approach to conserving the amphibian tree of life.},
author = {Jetz, Walter and Pyron, R. Alexander},
doi = {10.1038/s41559-018-0515-5},
issn = {2397334X},
journal = {Nature Ecology and Evolution},
title = {{The interplay of past diversification and evolutionary isolation with present imperilment across the amphibian tree of life}},
year = {2018}
}

@article{Bouckaert2014,
abstract = {We present a new open source, extensible and flexible software platform for Bayesian evolutionary analysis called BEAST 2. This software platform is a re-design of the popular BEAST 1 platform to correct structural deficiencies that became evident as the BEAST 1 software evolved. Key among those deficiencies was the lack of post-deployment extensibility. BEAST 2 now has a fully developed package management system that allows third party developers to write additional functionality that can be directly installed to the BEAST 2 analysis platform via a package manager without requiring a new software release of the platform. This package architecture is showcased with a number of recently published new models encompassing birth-death-sampling tree priors, phylodynamics and model averaging for substitution models and site partitioning. A second major improvement is the ability to read/write the entire state of the MCMC chain to/from disk allowing it to be easily shared between multiple instances of the BEAST software. This facilitates checkpointing and better support for multi-processor and high-end computing extensions. Finally, the functionality in new packages can be easily added to the user interface (BEAUti 2) by a simple XML template-based mechanism because BEAST 2 has been re-designed to provide greater integration between the analysis engine and the user interface so that, for example BEAST and BEAUti use exactly the same XML file format. {\textcopyright} 2014 Bouckaert et al.},
author = {Bouckaert, Remco and Heled, Joseph and K{\"{u}}hnert, Denise and Vaughan, Tim and Wu, Chieh Hsi and Xie, Dong and Suchard, Marc A. and Rambaut, Andrew and Drummond, Alexei J.},
doi = {10.1371/journal.pcbi.1003537},
issn = {15537358},
journal = {PLoS Computational Biology},
pmid = {24722319},
title = {{BEAST 2: A Software Platform for Bayesian Evolutionary Analysis}},
year = {2014}
}


@article{DinizFilho2012,
abstract = {Among the statistical methods available to control for phylogenetic autocorrelation in ecological data, those based on eigenfunction analysis of the phylogenetic distance matrix among the species are becoming increasingly important tools. Here, we evaluate a range of criteria to select eigenvectors extracted from a phylogenetic distance matrix (using phylogenetic eigenvector regression, PVR) that can be used to measure the level of phylogenetic signal in ecological data and to study correlated evolution. We used a principal coordinate analysis to represent the phylogenetic relationships among 209 species of Carnivora by a series of eigenvectors, which were then used to model log-transformed body size. We first conducted a series of PVRs in which we increased the number of eigenvectors from 1 to 70, following the sequence of their associated eigenvalues. Second, we also investigated three non-sequential approaches based on the selection of 1) eigenvectors significantly correlated with body size, 2) eigenvectors selected by a standard stepwise algorithm, and 3) the combination of eigenvectors that minimizes the residual phylogenetic autocorrelation. We mapped the mean specific component of body size to evaluate how these selection criteria affect the interpretation of non-phylogenetic signal in Bergmann's rule. For comparison, the same patterns were analyzed using autoregressive model (ARM) and phylogenetic generalized least-squares (PGLS). Despite the robustness of PVR to the specific approaches used to select eigenvectors, using a relatively small number of eigenvectors may be insufficient to control phylogenetic autocorrelation, leading to flawed conclusions about patterns and processes. The method that minimizes residual autocorrelation seems to be the best choice according to different criteria. Thus, our analyses show that, when the best criterion is used to control phylogenetic structure, PVR can be a valuable tool for testing hypotheses related to heritability at the species level, phylogenetic niche conservatism and correlated evolution between ecological traits. -� 2011 The Authors. Ecography -� 2012 Nordic Society Oikos},
author = {Diniz-Filho, Jose Alexandre F and Bini, Luis Mauricio and Rangel, Thiago Fernando and Morales-Castilla, Ignacio and Olalla-T{\'{a}}rraga, Miguel {\'{A}} and Rodr{\'{i}}guez, Miguel {\'{A}} and Hawkins, Bradford A},
doi = {10.1111/j.1600-0587.2011.06949.x},
isbn = {1600-0587},
issn = {09067590},
journal = {Ecography},
title = {{On the selection of phylogenetic eigenvectors for ecological analyses}},
year = {2012}
}


@misc{Santos2018,
author = {Santos, Thiago},
file = {:C\:/Users/adrie/AppData/Local/Mendeley Ltd./Mendeley Desktop/Downloaded/Santos - 2018 - Package ‘ PVR '. Phylogenetic Eigenvectors Regression and Phylogentic Signal-Representation Curve.pdf:pdf},
mendeley-groups = {R_packages},
title = {{Package ‘ PVR '. Phylogenetic Eigenvectors Regression and Phylogentic Signal-Representation Curve}},
year = {2018}
}

@article{Cadotte2011,
abstract = {1. The goal of conservation and restoration activities is to maintain biological diversity and the ecosystem services that this diversity provides. These activities traditionally focus on the measures of species diversity that include only information on the presence and abundance of species. Yet how diversity influences ecosystem function depends on the traits and niches filled by species. 2. Biological diversity can be quantified in ways that account for functional and phenotypic differences. A number of such measures of functional diversity (FD) have been created, quantifying the distribution of traits in a community or the relative magnitude of species similarities and differences. We review FD measures and why they are intuitively useful for understanding ecological patterns and are important for management. 3. In order for FD to be meaningful and worth measuring, it must be correlated with ecosystem function, and it should provide information above and beyond what species richness or diversity can explain. We review these two propositions, examining whether the strength of the correlation between FD and species richness varies across differing environmental gradients and whether FD offers greater explanatory power of ecosystem function than species richness. 4. Previous research shows that the relationship between FD and richness is complex and context dependent. Different functional traits can show individual responses to different gradients, meaning that important changes in diversity can occur with minimal change in richness. Further, FD can explain variation in ecosystem function even when richness does not. 5. Synthesis and applications. FD measures those aspects of diversity that potentially affect community assembly and function. Given this explanatory power, FD should be incorporated into conservation and restoration decision-making, especially for those efforts attempting to reconstruct or preserve healthy, functioning ecosystems.},
author = {Cadotte, Marc W. and Carscadden, Kelly and Mirotchnick, Nicholas},
journal = {Journal of Applied Ecology},
doi = {10.1111/j.1365-2664.2011.02048.x},
isbn = {0021-8901},
issn = {00218901},
keywords = {Biodiversity conservation,Community assembly,Ecosystem function,Ecosystem services,Functional diversity,Restoration,Species richness},
pmid = {1657255},
title = {{Beyond species: Functional diversity and the maintenance of ecological processes and services}},
year = {2011}
}

@article{Fox1992,
abstract = {Abstract Working in the context of the linear model y = X? + $\epsilon$, we generalize the concept of variance inflation as a measure of collinearity to a subset of parameters in ? (denoted by ? 1, with the associated columns of X given by X 1). The essential idea underlying this generalization is to examine the impact on the precision of estimation?in particular, the size of an ellipsoidal joint confidence region for ? 1?of less-than-optimal selection of other columns of the design matrix (X 2), treating still other columns (X 0) as unalterable, even hypothetically. In typical applications, X 1 contains a set of dummy regressors coding categories of a qualitative variable or a set of polynomial regressors in a quantitative variable; X 2 contains all other regressors in the model, save the constant, which is in X 0. If $\sigma$ 2 V denotes the realized variance of , and $\sigma$ 2 U is the variance associated with an optimal selection of X 2, then the corresponding scaled dispersion ellipsoids to be compared are ? v = {x : x?V ?1 x ≤ 1} and ? U = {x : x?U ?1 x ≤ 1}, where ? U is contained in ? v . The two ellipsoids can be compared by considering the radii of ? v relative to ? U , obtained through the spectral decomposition of V relative to U. We proceed to explore the geometry of generalized variance inflation, to show the relationship of these measures to correlation-matrix determinants and canonical correlations, to consider X matrices structured by relations of marginality among regressor subspaces, to develop the relationship of generalized variance inflation to hypothesis tests in the multivariate normal linear model, and to present several examples. Working in the context of the linear model y = X? + $\epsilon$, we generalize the concept of variance inflation as a measure of collinearity to a subset of parameters in ? (denoted by ? 1, with the associated columns of X given by X 1). The essential idea underlying this generalization is to examine the impact on the precision of estimation?in particular, the size of an ellipsoidal joint confidence region for ? 1?of less-than-optimal selection of other columns of the design matrix (X 2), treating still other columns (X 0) as unalterable, even hypothetically. In typical applications, X 1 contains a set of dummy regressors coding categories of a qualitative variable or a set of polynomial regressors in a quantitative variable; X 2 contains all other regressors in the model, save the constant, which is in X 0. If $\sigma$ 2 V denotes the realized variance of , and $\sigma$ 2 U is the variance associated with an optimal selection of X 2, then the corresponding scaled dispersion ellipsoids to be compared are ? v = {x : x?V ?1 x ≤ 1} and ? U = {x : x?U ?1 x ≤ 1}, where ? U is contained in ? v . The two ellipsoids can be compared by considering the radii of ? v relative to ? U , obtained through the spectral decomposition of V relative to U. We proceed to explore the geometry of generalized variance inflation, to show the relationship of these measures to correlation-matrix determinants and canonical correlations, to consider X matrices structured by relations of marginality among regressor subspaces, to develop the relationship of generalized variance inflation to hypothesis tests in the multivariate normal linear model, and to present several examples.},
author = {Fox, John and Monette, Georges},
doi = {10.1080/01621459.1992.10475190},
issn = {1537274X},
journal = {Journal of the American Statistical Association},
keywords = {Canonical correlation,Joint confidence regions,Variance inflation},
title = {{Generalized collinearity diagnostics}},
year = {1992}
}

 @Manual{DHARMa,
    title = {DHARMa: Residual Diagnostics for Hierarchical (Multi-Level / Mixed)
Regression Models},
    author = {Florian Hartig},
    year = {2021},
    note = {R package version 0.4.1},
    url = {https://CRAN.R-project.org/package=DHARMa},
  }
  
    @Article{performance,
    title = {{performance}: An {R} Package for Assessment, Comparison and Testing of Statistical Models},
    author = {Daniel Lüdecke and Mattan S. Ben-Shachar and Indrajeet Patil and Philip Waggoner and Dominique Makowski},
    year = {2021},
    journal = {Journal of Open Source Software},
    volume = {6},
    number = {60},
    pages = {3139},
    doi = {10.21105/joss.03139},
  }

@article{Dormann2013,
abstract = {Collinearity refers to the non independence of predictor variables, usually in a regression-type analysis. It is a common feature of any descriptive ecological data set and can be a problem for parameter estimation because it inflates the variance of regression parameters and hence potentially leads to the wrong identification of relevant predictors in a statistical model. Collinearity is a severe problem when a model is trained on data from one region or time, and predicted to another with a different or unknown structure of collinearity. To demonstrate the reach of the problem of collinearity in ecology, we show how relationships among predictors differ between biomes, change over spatial scales and through time. Across disciplines, different approaches to addressing collinearity problems have been developed, ranging from clustering of predictors, threshold-based pre-selection, through latent variable methods, to shrinkage and regularisation. Using simulated data with five predictor-response relationships of increasing complexity and eight levels of collinearity we compared ways to address collinearity with standard multiple regression and machine-learning approaches. We assessed the performance of each approach by testing its impact on prediction to new data. In the extreme, we tested whether the methods were able to identify the true underlying relationship in a training dataset with strong collinearity by evaluating its performance on a test dataset without any collinearity. We found that methods specifically designed for collinearity, such as latent variable methods and tree based models, did not outperform the traditional GLM and threshold-based pre-selection. Our results highlight the value of GLM in combination with penalised methods (particularly ridge) and threshold-based pre-selection when omitted variables are considered in the final interpretation. However, all approaches tested yielded degraded predictions under change in collinearity structure and the 'folk lore'-thresholds of correlation coefficients between predictor variables of |r| >0.7 was an appropriate indicator for when collinearity begins to severely distort model estimation and subsequent prediction. The use of ecological understanding of the system in pre-analysis variable selection and the choice of the least sensitive statistical approaches reduce the problems of collinearity, but cannot ultimately solve them. [ABSTRACT FROM AUTHOR]},
archivePrefix = {arXiv},
arxivId = {NIHMS150003},
author = {Dormann, Carsten F. and Elith, Jane and Bacher, Sven and Buchmann, Carsten and Carl, Gudrun and Carr{\'{e}}, Gabriel and Marqu{\'{e}}z, Jaime R.Garc{\'{i}}a and Gruber, Bernd and Lafourcade, Bruno and Leit{\~{a}}o, Pedro J. and M{\"{u}}nkem{\"{u}}ller, Tamara and Mcclean, Colin and Osborne, Patrick E. and Reineking, Bj{\"{o}}rn and Schr{\"{o}}der, Boris and Skidmore, Andrew K. and Zurell, Damaris and Lautenbach, Sven},
doi = {10.1111/j.1600-0587.2012.07348.x},
eprint = {NIHMS150003},
isbn = {1600-0587},
issn = {09067590},
journal = {Ecography},
pmid = {315892600003},
title = {{Collinearity: A review of methods to deal with it and a simulation study evaluating their performance}},
year = {2013}
}

@article{Cailliez1983,
abstract = {If d is a measure of dissimilarity on a finite set with n elements, the smallest positive constant c* such that d +c has an euclidean representation for all c ≥c* is shown to be the largest eigen-value of a matrix of size 2 n. {\textcopyright} 1983 The Psychometric society.},
author = {Cailliez, Francis},
doi = {10.1007/BF02294026},
issn = {00333123},
journal = {Psychometrika},
keywords = {additive constant,multidimensional scaling},
title = {{The analytical solution of the additive constant problem}},
year = {1983}
}





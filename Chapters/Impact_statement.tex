%% Impact statement -- no more than 500 words
As anthropogenic pressures on the world’s biota keep increasing, it is vital to put into place conservation measures to prevent and reverse further species loss. Beyond ethical and moral considerations, there is an urgent need to protect biodiversity because it sustains a range of ecological processes essential to human well-being and planetary health. Effectively managing biodiversity and related ecosystem processes in a changing world requires an understanding of how different species respond to anthropogenic disturbances. My thesis integrates various data sources to investigate the influence of traits on species land-use responses and on species climate-change sensitivity -- two of the most pressing threats on biodiversity -- at global scales and comparatively across the four terrestrial vertebrate classes. By asking whether interspecific trait variation is associated with species land-use responses and with climate-change sensitivity, my work consolidates our understanding of what renders species sensitive to environmental change, which can help prioritise conservation efforts. 
 
Chapter 2 presents a trait data collection for terrestrial vertebrates, targeting seven commonly-used traits. I highlight the global taxonomic, geographical, and phylogenetic biases in the trait data, revealing knowledge gaps which could guide future data collection efforts. Chapter 2 was published in \textit{Global Ecology and Biogeography}. The compiled data were made available and have since been used by researchers in the field (e.g., \citet{Capdevila2022b}) and downloaded 272 times as of May 2022.  Chapter 3 uses the collected trait data and reveals profound effects of land-use change on vertebrate functional diversity, which contributes to documenting global human impacts on vertebrates and also underlines the possible threats posed by land-use change to ecosystem processes sustained by vertebrates. Chapter 3 was published in \textit{Ecology Letters}. In Chapter 4, I ask whether traits are associated with species land-use responses and with species climate-change sensitivity, comparatively across the four vertebrate classes. Chapter 4 thus puts into perspective the usefulness of trait data for understanding how species respond to these anthropogenic changes, which is valuable for conservation planning and prioritisation. In Chapter 5, I ask whether species energetic requirements, estimated from metabolic rates, influence species persistence in disturbed land uses. Chapter 5 thus integrates physiological data to further our fundamental understanding of how vertebrate species respond to land-use change and of the potential consequences for ecosystem functioning. 

Beyond publishing two of my PhD Chapters, I have been able to disseminate my work at various international conferences (BES annual meetings in 2019, 2020 and 2021; BES Macroecology conference in 2019; IBS early-career conference in 2021). I will also present my PhD work at the IBS conference (June 2022, 10$^{th}$ Biennial meeting), and at the BES Macroecology conference (July 2022). I have contributed to the Living Planet Report 2020 \citep{WWF2020} and to other published papers \citep{Newbold2019, Newbold2020}. Overall, my PhD work consolidates our knowledge of the role of vertebrate traits for understanding species responses to human pressures and highlights the value of trait data, and more widely, of ecological knowledge, for preserving vertebrate species in a changing world.
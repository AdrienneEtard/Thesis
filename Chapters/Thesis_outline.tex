%%Thesis outline of contents, authorship and collaborations

\section*{Chapter 1: General introduction}
Chapter 1 presents the background for this thesis, exposes the fundamental concepts, and highlights the research questions I investigated in the different Chapters.

\section*{Chapter 2: Global gaps and biases in trait data for terrestrial vertebrates}
In Chapter 2, I present an analysis of the global gaps and biases in terrestrial vertebrate trait data. To this end, I collate data on seven traits commonly measured in terrestrial vertebrates. I then evaluate the availability of these trait data across the vertebrate classes, assessing whether there are taxonomic, phylogenetic and spatial biases. This chapter was published in \textit{Global Ecology and Biogeography} in 2020 (DOI: 10.1111/geb.13184; \citet{Etard2020}). The paper was co-authored by Sophie Morrill who collated some of the data on reptile traits as part of an MRes project at UCL, and by Tim Newbold, who participated in the development of the research questions, provided detailed feedback on the analyses, and contributed to the writing of the paper. 

\section*{Chapter 3: Intensive human land uses negatively affect vertebrate functional diversity}
In this Chapter, I investigate how land-use change affects the functional composition and functional diversity  of local vertebrate assemblages. This Chapter was published in \textit{Ecology Letters} in 2022  (DOI: 10.1111/ele.13926; \citet{Etard2022}) and co-authored by Alex Pigot and Tim Newbold, who helped construct the hypotheses, provided detailed feedback on the work, and took part in the writing of the paper.

\section*{Chapter 4: Geographical range area, habitat breadth and specialisation on natural habitats explain land-use responses and climate-change sensitivity more consistently than life-history and dietary traits in terrestrial vertebrates}
In this Chapter, I assess whether ecological traits as well as geographical range area are associated with species land-use responses and species estimated climate-change sensitivity, comparatively among terrestrial vertebrate classes. Rhiannon Osborne-Tonner contributed to this Chapter by collecting data on amphibian and reptile diet during her MSc project at UCL, which I used to complement my datasets. This Chapter was conducted in collaboration with Tim Newbold who helped develop the research questions and provided detailed feedback on the work and on the writing.  I plan to submit this Chapter as a research article to a scientific journal.

\section*{Chapter 5: Energetic constraints and trophic group explain species persistence in disturbed land uses}
In Chapter 5, I evaluate the impacts of land-use change on community energetic requirements, and I assess whether species energetic requirements influence species persistence in disturbed land uses. To this end, I use physiological data, compiling species resting metabolic rates (used as a proxy for energetic requirements) from the literature. Meghan Hayden and Laura Dee of the University of Colorado, Boulder, as well as Tim Newbold, contributed to the elaboration of the research questions for this Chapter. Meghan Hayden further contributed to this Chapter by retrieving information on net primary productivity for PREDICTS sites, using data from MODIS satellite imagery. All collaborators also provided feedback on the work and participated in writing the manuscript. This Chapter was submitted to a scientific journal and underwent a round of peer-review. I am preparing this Chapter for resubmission to a scientific journal.

\section*{Chapter 6: General discussion}
This final chapter summarises the main findings of my thesis and assesses their contributions to the field.
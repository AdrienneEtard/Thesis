%% Thesis general abstract

Human activities have profoundly impacted global biodiversity. Currently, anthropogenic land-use and climate change figure among the major threats to the world’s fauna. However, not all species respond similarly to these pressures. Interspecific variability in responses to human threats is notably underpinned by the fact that different species possess different attributes and intrinsic characteristics (traits), some of them allowing species to cope with environmental changes, while others confer a disadvantage to species in modified environments. Understanding what renders species sensitive to anthropogenic pressures is vital to inform and prioritise conservation efforts. Yet, in terrestrial vertebrates, a group for which ecological data is the most abundant, it remains unclear which traits are associated with higher sensitivity to human pressures.   The aims of my thesis are to investigate whether and which traits are associated with  land-use responses and climate-change sensitivity in terrestrial vertebrates, and to highlight some of the consequences for ecosystem functioning. I first assess the global availability of ecological trait data for terrestrial vertebrates, identifying understudied groups and regions (e.g., Central-African reptiles). I then show that, at global scales, disturbed land uses negatively impact the functional diversity of vertebrate assemblages.  Further, I find that in all classes, higher sensitivity to land-use and climate change is associated with  narrower ranges, smaller habitat breadth and inability to use human-modified habitats. Both land-use responses and climate-change sensitivity are unevenly distributed among dietary groups, highlighting potential food-web disruptions in assemblages under pressure.  Finally, I show that land-use responses are influenced by species' energetic requirements, so that energetic fluxes within vertebrate assemblages are likely modified under human-driven land-use change. Although the large-scale consequences of biodiversity changes for ecosystem functioning remain to be fully understood, my thesis highlights a compositional reshaping of vertebrate assemblages under human pressure and furthers our understanding of anthropogenic impacts on biodiversity.  
%% chapter 5 Metabolic  rates and energetic requirements

\section*{Keywords}
Land use; land-use intensity; metabolic rates; energetic constraints; terrestrial vertebrates; trophic level; occurrence.

\section*{Abstract}
Land-use change is the primary driver of global biodiversity loss. In terrestrial vertebrates, previous work has shown that sensitivity to land-use change depends on species traits, but the extent to which energetic constraints explain species responses to disturbed land uses remains largely unexplored. Here, we investigate relationships between the energetic requirements of terrestrial vertebrates (estimated from resting metabolic rates) and land-use change, at two levels of organisation. First, at the assemblage level we hypothesize that total energetic requirements in disturbed land uses are lower than in undisturbed land uses, assuming that there is less energy available in these areas. Second, after controlling for the effects of body mass and taxonomy on metabolic rates, we predict that species with relatively lower energetic expenditure are favoured over species with relatively higher energetic expenditure in disturbed land uses, as resource efficiency will be beneficial in these resource--poor environments. Because trophic level  can influence species ability to assimilate various types of food, we investigate whether our predictions are consistent among trophic levels (here, omnivores, carnivores or herbivores). 

Our results challenged both our hypotheses. We found that total assemblage-level energetic requirements did not systematically decrease in disturbed land uses. For instance, we detected significant increases for urban areas in all trophic levels. Second, we found a positive effect of metabolic rates (after controlling for body mass and taxonomy) on species probability of occurrence across all trophic levels for at least one of the most disturbed land uses we consider (pasture, cropland and urban). Species with higher energetic expenditure may display a set of characteristics rendering them more able to cope with disturbances, such as higher activity levels or bigger brain sizes. Our findings highlight that land-use change has significant impacts on vertebrate community metabolism. 

\section{Introduction}

\section{Methods}

\section{Results}

\section{Discussion}

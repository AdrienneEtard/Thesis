%% chapter 5 Metabolic  rates and energetic requirements

\section*{Keywords}
Land use; land-use intensity; metabolic rates; energetic constraints; terrestrial vertebrates; trophic level; occurrence.

\section*{Abstract}
Land-use change is the primary driver of global biodiversity loss. In terrestrial vertebrates, previous work has shown that sensitivity to land-use change depends on species traits, but the extent to which energetic constraints explain species responses to disturbed land uses remains largely unexplored. Here, I investigate relationships between the energetic requirements of terrestrial vertebrates (estimated from resting metabolic rates) and land-use change, at two levels of organisation. First, at the assemblage level, I hypothesize that total energetic requirements in disturbed land uses are lower than in undisturbed land uses (assuming that there is less energy available in these areas). Second, after controlling for the effects of body mass and taxonomy on metabolic rates, I predict that species with relatively lower energetic expenditure are favoured over species with relatively higher energetic expenditure in disturbed land uses, as resource efficiency may be beneficial in resource--poor environments. Because trophic level can influence species ability to assimilate various types of food, I investigate whether my predictions are consistent among trophic levels (here, omnivores, carnivores or herbivores). 

The results challenged both hypotheses. I found that total assemblage-level energetic requirements did not systematically decrease in disturbed land uses. For instance, I detected significant increases for urban areas in all trophic levels, highlighting that disturbed areas may not be as energy--poor as I initially assumed. Second, I found a positive effect of metabolic rates (after controlling for body mass and taxonomy) on species probability of occurrence across all trophic levels for at least one of the most disturbed land uses I consider (pasture, cropland and urban). Species with higher energetic expenditure may display a set of characteristics rendering them more able to cope with disturbances, such as higher activity levels or bigger brain sizes, which could help them make use of the resources available in disturbed environments. The findings of this Chapter highlight that land-use change has significant impacts on vertebrate community metabolism. 

\section{Introduction}
  

\section{Methods}

\subsection{Vertebrate assemblage composition }

I obtained vertebrate assemblage composition in different land uses from the PREDICTS database (Hudson et al., 2014, 2017). The PREDICTS database is a large collection of published studies that measure biodiversity across different land uses and is one of the most comprehensive global databases of its type. In each PREDICTS study, species occurrence and often abundance were recorded across different sites. Each site was assigned to one of the following land-use types: primary vegetation (natural, undisturbed vegetation), secondary vegetation (recovering after complete destruction of primary vegetation), plantation forest (woody crops), pasture (areas grazed by livestock), cropland (herbaceous crops) and urban (built-up areas). The land-use categories were assigned based on habitat descriptions from the original studies (Hudson et al. (2014), sometimes in consultation with the original study authors. Each site was also classified in terms of land-use intensity as either minimal, light or intense. The land-use-intensity assignment was also made on the basis of the habitat description in the original studies, and depended on criteria specific to each land use (such as degree of mechanisation, yield or chemical inputs for cropland; or the amount of green space in urban areas; Hudson et al. (2014)).  

I subset the PREDICTS database for studies that sampled terrestrial vertebrates, and for which both land use and land-use intensity had been characterised. I thus obtained 181 studies for 4,238 species sampled across 6,484 sites (Fig. 1a). Sample sizes varied across land uses and land-use intensities (Fig. 1b). 

\subsection{Energy availability by land-use type and land-use intensity}

The predictions of this Chapter rely on the assumption that there is less energy available in disturbed compared to undisturbed land uses. To test this assumption, I used terrestrial net primary productivity (NPP) across land uses as a proxy for available energy. NPP quantifies the amount of atmospheric carbon fixed by plants and accumulated as biomass. I derived NPP using imagery from the Moderate Resolution Imaging Spectroradiometer (MODIS) on board NASA’s Terra satellite. NPP estimates are based on a yearly composite of measures made at 8-day intervals, captured at 500-m spatial resolution (Running \& Zhou, 2015). NPP was obtained for 4,062 of the PREDICTS database sites used in the analysis (I matched the sites to the NPP data using the sampling year available in PREDICTS). I fit a linear mixed-effects model (lme4 package, version 1.1-23, Bates et al. (2015)) explaining site-level NPP by land use and land-use intensity, with a random intercept accounting for study identity, to control for differences in experimental design across studies. Model predictions showed that NPP decreased significantly in several land uses (e.g., pasture and cropland) compared with the primary vegetation reference level, although the strength and in some cases direction of the difference varied among land-use and intensity combinations (e.g., increases in urban land uses; Fig. 1c).

\subsection{Resting Metabolic Rates (RMR) \& imputations of missing RMR values}

As a proxy for species-level energetic expenditure, I used estimates of the minimum amount of energy required for organismal maintenance, i.e., basal metabolic rates (BMR) for endotherms, and resting metabolic rates (RMR) for ectotherms. From the literature, I obtained estimates of BMR for 719 species of birds and 685 mammals, and estimates of RMR for 126 amphibians and 173 reptiles (Supporting information, Table S1). For endotherms, BMR are measured when species are in their thermoneutral zone, that is, when there is little to no energy expenditure allocated to thermoregulation. Thus, BMR estimates were derived from lab studies that mostly measured oxygen consumption of the organisms at rest under controlled conditions and in the thermoneutral zone of the species. For an ectotherm, there is no `basal' metabolic rate, as body temperature mainly depends on environmental temperature. Their metabolic rates follow a hump-shaped relationship with environmental temperature, highest at an optimal temperature which corresponds to a performance peak. To be able to compare endotherms’ BMR with ectotherms’ RMR, Stark et al. (2020) used the metabolic rates that correspond to a performance peak for both groups (i.e., BMR in the thermoneutral zone for endotherms, and metabolic rates at optimal temperature for ectotherms). Thus, I used the data compiled in Stark et al. (2020) for ectotherms, and from the sources specified in Table S1 for endotherms. The units for BMR and RMR were standardized to mL of dioxygen consumed per hour (mLO$_2$/h). As in Stark et al. (2020), I henceforth refer to both basal and resting metabolic rates as RMR. 

For the species occurring in PREDICTS, initial data coverage for RMR was poor (Table S1), necessitating imputation of missing values. To do so, I first measured the phylogenetic signal in BMR and RMR (log$_e$-transformed), using Pagel’s $\lambda$ (Pagel, 1999), to assess whether metabolic rates were sufficiently phylogenetically conserved to be estimated from species phylogenetic position. I obtained class-specific phylogenetic trees from Jetz et al. (2012) for birds, from Faurby et al. (2018, 2020) for mammals, from Tonini et al. (2016) for reptiles (squamates), and from Jetz \& Pyron (2018) for amphibians (all downloaded in April 2020). For each class, I randomly sampled 100 trees. To account for phylogenetic uncertainty, I calculated Pagel’s $\lambda$ for each sampled tree and reported the median value, as well as the 2.5th and 97.5th percentiles (Table S1).  

In addition to being highly phylogenetically conserved (Table S1), RMR correlate strongly with body mass (Fig. 2a). Thus, I imputed missing values using body-mass information (see next section), phylogenetic relationships and taxonomic orders as predictors (Penone et al. 2014). For each class, I used a consensus phylogenetic tree from which I summarised phylogenetic relationships in the form of five phylogenetic eigenvectors. Including more eigenvectors had little impact on the imputed values (results not shown). 
\begin{comment} For instance, after re-imputing the values using 10 eigenvectors, the correlation coefficient between the values imputed with 5 eigenvectors and with 10 eigenvectors was 0.997, showing high levels of congruence between the imputed values, regardless of the number of eigenvectors). 
\end{comment}
Consensus trees were obtained with the TreeAnnotator programme of the BEAST software (Bouckaert et al. 2014). Missing RMR values were imputed using random forests algorithms implemented in R using the ‘missForest’ package (Version 1.4; Stekhoven \& Bühlmann 2012; Penone et al. 2014; Stekhoven 2016).  

\subsection{Trophic level and body mass information}

I used body mass and trophic level information for terrestrial vertebrates compiled in Chapter 2. Body mass was compiled as a single measure at the species level, meaning I was unable to consider intraspecific variation. Trophic level described species as either carnivores, omnivores, or herbivores. Because there were gaps in the availability of the data, more so for trophic level than for body mass (see Chapter 2), I imputed the missing trait values (independently of RMR imputations), then used both imputed and empirical body mass values for imputations of missing RMR values. To impute missing body mass and trophic levels, I used random forests algorithms (using the missForest R package; Stekhoven \& Bühlmann (2012); Stekhoven (2016)), including as additional predictors phylogenetic information, added in the form of 10 phylogenetic eigenvectors (Diniz-Filho et al. 2012) following Penone et al. (2014), and also taxonomic order. I considered a wider set of life-history traits in the missing values imputations: lifespan, litter/clutch size, habitat breadth and use of artificial habitats (also compiled in Chapter 2). Phylogenetic eigenvectors were extracted from the class-specific phylogenies using the PVR package (Santos 2018).  

\subsection{Effects of land use, land-use intensity and trophic level on assemblage-level total RMR (prediction 1; Figure 2(a))}

Assemblage-level total RMR (tRMR) was obtained by summing abundance-weighted RMR for the species occurring in each site; abundance data were available for 125 of the 181 PREDICTS studies I considered (sampling 3,487 species across 4,644 sites). I fitted a linear mixed-effects model to explain log$_e$-tRMR as a function of land use, land-use intensity and trophic level, with a random intercept accounting for study identity to control for differences in experimental design across studies. I started with a model allowing all two-way interactions among the predictors. I then tested whether adding the three-way interaction among land use, land-use intensity and trophic level improved the fit of the model, using a likelihood-ratio test. The model that included the three-way interaction was retained (P << 0.01; model 1, Fig. 2). In addition, because it is well established that resting metabolic rates are influenced by temperature (Clarke \& Fraser 2004), I checked whether including annual mean temperature in the model affected the conclusions. Annual mean temperature at each PREDICTS site was estimated from WorldClim version 2.1 (Fick \& Hijmans 2017), using a 2.5 arc-minute resolution. Adding annual mean temperature did not improve model fit (likelihood-ratio test: P = 0.113), thus I did not consider its effects any further.   

\subsubsection*{Model validation.}
To ensure that imputation uncertainty did not affect the conclusions, I refitted model 1 using the subset of species (n = 426) from PREDICTS for which there were empirical RMR information (i.e., excluding imputed RMR values). 

\subsubsection*{Disentangling the effects of body mass and abundance on tRMR.}
Since RMR correlates strongly with body mass, changes in tRMR are likely to be driven in part by changes in the size-spectrum of ecological assemblages. I fitted an additional model to explain changes in species’ abundance (given presence) by land use, land-use intensity, trophic level, body mass and their interactions, to understand the role of shifts in the body of species on observed changes in tRMR (see Fig. S1).  

\subsection{Effects of land use, land-use intensity, trophic level and residual RMR on species occurrence probability (Prediction 2; Figure 2(b))}

To control for the effects of body mass and taxonomy on RMR, I used the residual variation in RMR after accounting for these variables, from a linear mixed-effects model fitting log$_e$-RMR as a function of log$_e$-body mass with nested random taxonomic effects (1|Class/Order/Family; Fig. 2). Hence, I used a metric that describes how much more energy (positive deviations) or less energy (negative deviations) than expected from body mass and taxonomic position a species spends for organismal maintenance. Similar approaches have been used in previous papers (Furness \& Speakman 2008; Naya et al. 2013). As detailed earlier, I expect species with lower residual RMR to do better in disturbed land uses than species with higher residual RMR (prediction 2; Fig. 2b) because, given any body mass, investing less energy in maintenance could contribute to persistence in a context of resource scarcity. 

To test the second prediction, I fitted a binomial mixed-effects model explaining species occurrence with land use, land-use intensity, trophic level and residual RMR. I started with a complete model that included all two-way interactions among the main effects. Because I wanted to test whether the second prediction was valid for each trophic level, I needed to account for potential differences in the slope of the relationships between occurrence probability and residual RMR among trophic levels. Thus, I performed a forward stepwise selection procedure to test whether adding three-way interactions among (1) land use, trophic level and residual RMR and (2) among land-use intensity, trophic level and residual RMR improved model fit, using likelihood-ratio tests. The final model included both three-way interactions (Fig. 2b; model 2). I fitted random effects that accounted for species identity, as well as for study and site identity within PREDICTS.  

\subsubsection*{Model validation.}
I checked the phylogenetic signal in the model residuals using Pagel’s $\lambda$ (Pagel 1999). Non-significant phylogenetic signal in the residuals would indicate that fitting species identity in the model’s random effects was sufficient to account for residual phylogenetic variation in RMR. Further, to assess the potential effects of imputation uncertainty on the results, I again fitted model 2 on the data subset for the 489 species with collected empirical RMR values, across 5,948 sites in 151 studies (i.e., excluding imputed values).  


\section{Results}

\section{Discussion}

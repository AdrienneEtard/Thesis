%% Chapter 6: General Discussion

In recent centuries, large-scale anthropogenic modifications to the Earth's systems have accelerated and reached unprecedented levels \citep{Steffen2015}. Two of the major signatures of human impacts on the Earth's systems are the transformation of the land surface \citep{Ellis2010}, notably fuelled by the rising demand for agricultural goods, and anthropogenic climate change onset by human-driven modifications to atmospheric composition \citep{Lewis2015}. Such transformations have had important impacts on the world's biota. By modifying species' habitats at large scales and altering local climatic conditions at rates exceeding natural variability, human changes have put biodiversity under pressure \citep{Maxwell2016}. Empirical evidence showing the extent and magnitude of human impacts on biodiversity has been accumulating \citep{Newbold2015, Young2016, Daru2021}. In particular, land-use and climate change have already driven declines in species richness and abundance, and altered species distributions, phenology and physiology \citep{Portner2008, Chown2010, Chen2011, Dirzo2014, Lenoir2015, Newbold2015, Soroye2020, Inouye2022, Butchart2010}, and increased extinctions rates to unprecedented levels \citep{Barnosky2011, Ceballos2015, DeVos2015} -- in some cases putting hundreds of millions of years of evolution at risk \citep{Nowakowski2018a, IUCN2020}. As anthropogenic pressures on biodiversity are unlikely to reduce given current scenarios of human development \citep{Stehfest2019}, and as international targets aiming to protect biodiversity and related ecosystem services have failed to be met \citep{Buchanan2020}, it is vital that we keep pursuing conservation and mitigating efforts to minimise or even reverse human impacts on biodiversity.

Human pressures impact species unevenly; for instance, past work has highlighted phylogenetic and spatial biases in species vulnerability to human pressures \citep{Fritz2009, Yessoufou2012, Ducatez2017, Weeks2022}. Some species (termed `winners' in past work) may benefit from global changes, while other species (termed `losers') are likely to decline. Understanding the factors that underpin interspecific variation in responses to human pressures is valuable to conservation planning, as it can help target and prioritise species at most risk from different threats. One of the reasons why species differ in their responses to environmental change is that species are  possess different characteristics, or traits. In my thesis, following \citet{McGill2006}, I defined traits as characteristics measurable at the level of an individual (i.e., intrinsic), comparable across different species, with likely impacts on organismal fitness or performance. Asking whether species traits relate to species' responses to land-use and climate change can thus help understand interspecific differences  in responses to human threats \citep{Munstermann2021}, and may help assess which species are more likely to be `winners' or `losers' under particular threatening processes. However, past studies that have tackled this question in terrestrial vertebrates have often been limited in both taxonomic and spatial coverage, so that it remains unclear whether there are general patterns in trait-sensitivity associations with human pressures across vertebrate species. Further, comparative studies of the sensitivity to both land-use and climate change among terrestrial vertebrate classes have been lacking. Understanding whether different human pressures are likely to favour similar species across diverse taxonomic groups could help conservation efforts. In my thesis, I aimed to fill in these gaps, investigating whether and which traits are associated with land-use responses and climate-change sensitivity comparatively across terrestrial vertebrates  and at global scales. My thesis also aimed to highlight some of the possible consequences for ecosystem functioning. 

To this end, I started by compiling trait data across terrestrial vertebrates, and by assessing the gaps and biases in the availability of trait data (Chapter 2). Bringing together these compiled data and a database containing species records in different land-use types (PREDICTS), I assessed the effects of land use and land-use intensity on the functional diversity of local terrestrial vertebrate assemblages (Chapter 3). I then asked whether species ecological characteristics (in which I included species ecological traits and species geographical range area ) were associated with species' land-use responses and with species' estimated climate-change sensitivity, comparatively across the four vertebrate classes (Chapter 4). Finally, I investigated the effects of land use and land-use intensity on the total energetic requirements of vertebrate assemblages , and I further assessed whether species' energetic requirements influenced species' land-use responses (Chapter 5). In this final Chapter, I synthesise the key findings of my work, and I assess their contributions to the broader knowledge and their relevance for the field. I highlight some of the limitations of my work and further challenges, notably reflecting on the current challenges to the application of trait-based approaches at large scales in animal taxa. 

\section{Gaps and biases in the knowledge of terrestrial vertebrates}

Although terrestrial vertebrates have been extensively studied in the past, there remain important gaps in our knowledge. Some of these gaps are illustrated in Chapter 2, which demonstrates the biases in the availability of ecological trait data (i.e., the ‘Raunki{\ae}ran’ shortfall \citep{Hortal2015}). After collating data for seven commonly-used ecological traits, I showed that the sampling of these traits  presented taxonomic, phylogenetic and spatial biases. Mammals and birds were overall well sampled for most traits (with a median trait coverage of 85\% for birds, and of 95\% for mammals). However, amphibians and reptiles presented acute gaps, with a 32\% median coverage for amphibians, and 38\% for reptiles. Chapter 2 further showed that such gaps were non-randomly distributed with regards to species phylogenetic position, with certain clades being under-sampled compared to others (for example, the family \textit{Ranidae}, or true frogs ; or the blind snakes of the family \textit{Typhlopidae}, and the  worm snakes of the family \textit{Amphisbaenidae}). Hence, Chapter 2 showed that amphibians and reptiles are understudied compared to mammals and birds. Knowledge gaps are thus acute for the most diverse vertebrate class (reptiles) as well as for the most threatened vertebrate class (amphibians; \citet{IUCN2020}).
 
Chapter 2 also highlights that knowledge gaps for amphibians and reptiles were non-randomly distributed across geographical space. For instance, the availability of trait data was significantly positively associated with species richness in several biogeographic realms (e.g., in the Australasian realm for reptiles, and in the Neotropics for amphibians), and significantly negatively associated with species richness in other realms (e.g., in the Indo-Malayan realm for both amphibians and reptiles). Chapter 2 thus highlighted some critically under-sampled regions for traits in amphibians and reptiles (e.g., the Congo-Basin). As discussed in Chapter 2, such geographical biases may be driven by uneven primary data collection efforts, themselves possibly explained by interacting socioeconomic factors \citep{Collen2008, Martin2012, Hortal2015, ONU2015}. Importantly, Chapter 2 shows that trait information may be less available in some of the most species-rich regions, critically important for global biodiversity conservation \citep{Barlow2018}. Chapter 2 further showed that, in all classes, the availability of trait information depended on species' rarity (captured by geographical range area), with more-widely distributed species being on average better sampled for traits than less widely distributed species. Such trends could be explained by a sampling bias towards more easily detectable species, and is concerning as narrow-ranging species have been shown to be at higher risks of extinction \citep{Chichorro2019}. 

It is worth highlighting that I targeted traits that are commonly used by ecologists in Chapter 2; knowledge gaps are likely to be even more important when looking at other traits, that are maybe less frequently used (such as dispersal abilities, which are available for a small fraction of vertebrate species only, despite their likely importance for understanding species' responses to anthropogenic pressures; \citet{Schloss2012, Lenoir2015}). Chapter 4 and Chapter 5 provide with an illustration of this point: in Chapter 4, no information about reptile diet could be obtained from an existing published database; while in Chapter 5, the availability of physiological data on resting metabolic rates was limited to a small number of species, even in mammals and birds. Further, lack of available data prevented me from considering intraspecific variation in my thesis, despite its likely importance for understanding species- and assemblage-level responses to human-driven change \citep{Carlson2014, Guralnick2016, Rohr2018}. 

The `Raunkiæran' shortfall, coupled with the lack of a centralised repository for animal trait data, may have been a major hindrance to the application of trait-based approaches at large scales in vertebrate species. In my thesis, I employed imputation techniques throughout to estimate missing trait values. However, complementing existing datasets with empirical data is the only way to fill the current data gaps robustly. It may be that integrating regional databases, potentially in different languages, could help fill some of the gaps.

Finally, many studies have called for the integration and standardisation of trait databases \citep{Kissling2018, Schneider2019, Weiss2019, Junker2022}, to produce open-access, interoperable, and unified products. Developing an appropriate framework for data standardisation and integration would require overcoming implementation and conceptual difficulties, notably pertaining to the definition of comparable traits across organisms and to the integration of taxonomic nomenclatures \citep{SalgueroGomez2021}. However, integration and standardisation of trait databases would represent a major step for animal ecology, preventing the duplication of research efforts, promoting collaboration and opening new research avenues. By highlighting the current gaps in terrestrial vertebrate trait data, Chapter 2 could help guide future collection efforts. 


\section{Functional reshaping of local vertebrate assemblages under land-use change}

In Chapter 3 and Chapter 5, my thesis highlights two different dimensions of the functional reshaping of vertebrate assemblages under land-use change. While Chapter 3 focuses on the functional diversity of local vertebrate assemblages by using various indices to summarise the diversity of ecological traits across vertebrate assemblages of the PREDICTS database, Chapter 5 makes use of physiological data to quantify energetic requirements at the assemblage level. To my knowledge, my work constitutes the first global assessment of the responses of functional diversity to land use and land-use intensity across vertebrate classes, and of the changes in total vertebrate energetic requirements with land use and land-use intensity.

\subsection{Functional diversity and functional composition (Chapter 3)}

In Chapter 3, after combining the trait data compiled in Chapter 2 with the PREDICTS database, I investigated how land use and land-use intensity affects the functional diversity (i.e. , functional richness and functional dispersion) as well as the functional composition (i.e., functional loss and functional gain) of terrestrial vertebrate assemblages, across and within vertebrate classes, at global scales. I investigated potential spatial variation in responses by looking for differences between temperate and tropical areas. The findings of this Chapter are threefold. First, across all vertebrate species, the functional diversity of vertebrate assemblages is negatively impacted by human land uses. Both functional richness and functional dispersion showed important average decreases in some disturbed land uses (e.g., a 63\% average decline in functional richness in intensely-used tropical cropland; a 20\% average decline in intensely-used urban areas). Further, decreases in functional dispersion exceeded the decreases expected from species loss in a number of disturbed land uses (most notably in tropical cropland), providing evidence of functional clustering of vertebrate assemblages in these land uses. Second, Chapter 3 highlights the spatial and taxonomic variation in responses: overall, functional richness tended to be more negatively affected in the tropics than in temperate areas (but responses were similar between the two areas for functional dispersion). The functional diversity of amphibians and birds was overall more negatively impacted than that of mammals and reptiles. Finally, Chapter 3 showed that vertebrate assemblages in disturbed land uses were subject to high levels of functional loss, and in some cases functional gain, highlighting important functional turnover in vertebrate assemblages. 

Chapter 3 thus highlights that the functional composition of vertebrate assemblages is reshaped in disturbed land uses. My results are mostly  line with previous studies that have been more taxonomically or geographically restricted \citep{Flynn2009, Matuoka2020, Marcacci2021}. Although the functional diversity of tropical areas emerges as being more sensitive than that of temperate areas, I also found important decreases in functional diversity in temperate areas, a pattern that had not been detected in past work \citep{Matuoka2020}. My results indicate that land-use change is a potential threat to ecosystem processes, in particular those sustained by species falling in sensitive areas of trait space. Further, losses of functional diversity in tropical areas are particularly problematic given that those areas are more species-rich than temperate areas, harbour many biodiversity hotspots and protected areas, and are currently subjected to some of the highest rates of land-use change \citep{Laurance2012, Hansen2013, Spracklen2015}. 

\subsection{Energetic requirements (Chapter 5)}
In Chapter 5, I combined the PREDICTS database with species-level estimates of resting metabolic rates. I investigated whether the total energetic requirements of vertebrate assemblages differed among different land-use types and trophic groups (classified as omnivores, herbivores, and carnivores). Contrary to my expectations, I found that the minimum amount of energy required by vertebrate assemblages did not show systematic decreases in disturbed land-use types. Across all three trophic groups, total energetic requirements (estimated from total abundance-weighted resting metabolic rates) even showed strong increases in some disturbed land uses (e.g., an average increase of 200\% in lightly-used urban areas for carnivores). These findings highlight that disturbed land uses have significant impacts on ecosystem functioning and emphasize that land-use change may promote significant changes to the local energetic balance of vertebrate assemblages. To my knowledge, this work constitutes the first direct global quantification of the impacts of land-use and land-use intensity on vertebrate energetic requirements. My findings are tightly linked to studies that have investigated changes in the body mass composition of vertebrate assemblages under land-use change (e.g., \citet{Newbold2020}), and also to studies drawing from food-web theory (e.g, impacts of habitat loss and fragmentation on the body size distribution in a trophic chain; \citet{Hillaert2020}). 

\vspace{0.5cm}
Altogether, the findings of Chapter 3 and 5 show that land-use change reshapes the functional composition of vertebrate assemblages. Land-use change may reduce native ecological trait diversity, constricting used areas of the trait space, thus potentially disrupting ecosystem processes sustained by the native species that are located in sensitive areas of the trait space . Further work could investigate the contributions of non-native species to the functional reshaping of local vertebrate assemblages. Changes in species composition in disturbed land uses have consequences for ecosystem functioning, notably significantly impacting the amount of energy locally processed by vertebrate species.


\section{Uneven sensitivity of vertebrate species to land-use and climate change}

In Chapter 4 and Chapter 5, I investigated associations between species-level characteristics and species' sensitivity to land-use and climate change. Chapter 4 focuses on ecological characteristics (ecological traits and geographical range area), and to my knowledge constitutes the first work to investigate associations between species' ecological characteristics and two human pressures, at global scales and comparatively across vertebrate classes. Chapter 5 uses physiological data and focuses on species' energetic requirements. To my knowledge, there has yet not been a study investigating whether species' energetic requirements influence species' responses to land use and land-use intensity in vertebrate species.

\subsection{Ecological characteristics (Chapter 4)}

In Chapter 4, I complemented the trait data from Chapter 2 with species-level dietary information and geographical range area. On the one hand, I combined these species-level data with the PREDICTS database, and I investigated how species' ecological characteristics influenced their responses to land use and land-use intensity . On the other hand, after estimating climate-change sensitivity from properties of species' climatic niche space, I investigated the associations between species' ecological characteristics and climate-change sensitivity. First, I found that narrower ranges, smaller habitat breadth, and inability to exploit artificial habitats were consistently associated with more negative land-use responses and higher climate-change sensitivity across vertebrate classes. Second, the associations of other traits was both class- and pressure-dependent. Overall, invertebrate eaters and fruit/nectar eaters tended to be negatively affected in disturbed land uses in all classes; in addition, invertebrate- and plant/seed-eating birds had higher climate-change sensitivity.

Chapter 4 thus highlights that both land-use and climate change are likely to favour similar species, that is, wider-ranging species, those with larger habitat breadth and those able to use artificial habitats. My work aligns with previous analyses on extinction risks \citep{Chichorro2019} and climate-change responses \citep{MacLean2017}, and shows that similar species might be favoured by both these human pressures. Further, the narrow-ranging species, which are on average less well known for trait data  (as shown in Chapter 2) and potentially for many other aspects of their biology, are those that are more likely to be more sensitive to human pressures. Such species may however make distinct and unique contributions to ecosystem functioning \citep{Mouillot2013,Dee2019}.  Further investigations of the ecosystem processes sustained by these geographically rarer species and their contributions to ecosystem functioning may be helpful to the mitigation of human impacts. 

I would like to underline some of the limitations of this work: first, as discussed in Chapter 4, I considered land-use change and climate change separately, thus not accounting for potentially interactive effects between these pressures. However, human pressures are likely to interact, and combined effects of land-use and climate change could have more deleterious (synergetic) effects than individual pressures acting independently \citep{Williams2020a, Williams2022}. Second, my findings rely on the use of the PREDICTS database (in this Chapter and in Chapters 3 and 5) and on the use of geographical distributions, and as such I used a `space-for-time' approach  to infer land-use responses and climate-change sensitivity from spatial data \citep{Blois2013}. Thus, I assumed that vertebrate assemblages and current species distributions were at equilibrium \citep{DePalma2018}, and I did not consider potential recovery effects, long-term population declines or any other temporal dynamics. However, it is likely that populations are not at equilibrium \citep{DePalma2018, Damgaard2019}. For instance, \citet{Sales2022} showed that range contractions of megafaunal mammals over the Late Pleistocene could lead to an underestimation of the realised climate niches of these species when using current distribution data, so that the estimated climate-change sensitivity of these species may be overestimated \citep{Sales2022}. Further, biotic lags and delays in biodiversity responses emphasise the need to account for land-use history \citep{DePalma2018, LeProvost2020}. Using long-term species records and population data may provide insights into the long-term effects of land-use and climate change on vertebrate species. 

Finally, in Chapter 4, I conducted a correlative assessment of the associations between species ecological characteristics and species' responses to land use and species' climate-change sensitivity. I found a number of significant associations, but traits had overall low explanatory power in the models, thus putting into question the degree to which traits can be used to infer species' responses to environmental change. Further, my results do not allow for mechanistic interpretations of how traits influence species' responses to land-use and climate change. Using long-term population data, mechanistic models and \textit{in silico} and field experiments may help uncover some of these mechanisms \citep{Ries2004, Boult2021}.

\subsection{Mass-independent energetic requirements (Chapter 5)}

Chapter 5 highlights the effects of the interactions between species' energetic requirements and trophic group on responses to land use and land-use intensity (after controlling metabolic rates for the effects of body mass and taxonomy, which explain a large proportion of the interspecific variation in metabolic rates). The key finding from this analysis contradicted my initial prediction: in all trophic groups, species with larger energetic expenditure (relative to body mass and taxonomy)  tended to do better in some of the most disturbed land-use types, compared to species with lower energetic expenditure. Thus, my work highlights that land-use change may favour species that have higher rates of mass-independent energy consumption, which could be linked with interspecific differences in behaviour, as discussed in Chapter 5 in more details. Species with larger energetic expenditure may display a set of characteristics that render them better able to cope with anthropogenic disturbances, such as higher activity levels \citep{Biro2010, Coogan2018}. Such species may also have a larger brain (a metabolically more consuming organ per unit mass; \citet{Isler2006}), which could play a significant role on species' ability to persist in disturbed areas \citep{Sayol2020}.

Finally, I would like to emphasise that my work gives important indications of how human pressures may affect ecosystem functioning. For example, the higher sensitivity of invertebrate eaters to both land-use and climate change may indicate that the processes that such species underpin, such as pest control, might be put at risk  \citep{Civantos2012}. However, we lack large-scale quantifications and empirical measurements of ecosystem processes sustained by vertebrates \citep{Wenny2011, Luck2012}. Linking vertebrate traits to particular ecosystem functions at large scales remains challenging, maybe because of a lack of empirical data, but also maybe because ecosystem processes sustained by vertebrate species are difficult to quantify (potentially because of the mobility of these species). Vertebrates nevertheless contribute significantly to ecosystem functioning \citep{Wandrag2015, Breviglieri2017, Ratto2018, Zhang2018_trophicinter}, emphasising the need to mitigate human impacts on them.


\section{Relevance of the findings to conservation}
Overall, my findings show that land-use and climate change may favour similar vertebrate species: `winners' are likely to be wider-ranging species that are able to occupy diverse habitats, including human-disturbed habitats; `winners' might also be able to allocate more energy to sustain higher activity levels. On the other hand, `losers' are likely narrow-ranging species, with natural habitat specialism and unlikely to be able to persist in human-disturbed areas. Thus, considerations of geographical rarity and specialisation indices appear to be highly relevant for characterising species at risk from both land-use and climate change when working at large scales and across all terrestrial vertebrates. In fact, such criteria have been employed by the IUCN Red List for assessing species' extinction risk \citep{Rodrigues2006}, as well as in predictive vulnerability assessments (e.g., F\citet{Foden2013}). Further, considerations of species' geographical rarity and specialisation can be useful to species-based as well as area-based conservation prioritization, for instance for identifying key areas for biodiversity conservation and for targeting species of interest \citep{Asaad2017, Mace2006}. My results thus lend support to the idea that vulnerability assessments and hierarchisation frameworks should take geographical rarity and specialisation into account \citep{LeBerre2019}.


\section{Conclusion}
My thesis constitutes, to my knowledge, the first attempt to use trait-based approaches at global scales to investigate associations between sensitivity to human pressures and species traits, comparatively across the terrestrial vertebrate classes and human pressures (land-use and climate change). I demonstrated that there exist major gaps and biases in our global ecological knowledge of terrestrial vertebrate species, particularly affecting amphibians and reptiles and some of the most species-rich regions, which is problematic given the higher sensitivity of these areas to human pressures. My work indicates that land-use and climate change are reshaping vertebrate biodiversity. I highlighted two dimensions of such functional reshaping, using ecological characteristics on the one hand, and physiological data on the other hand. Land-use and climate change may tend to favour species that are wider-ranging, have larger habitat breadth and are able to use artificial habitats; and land-use change alone tends to favour species that are able to allocate more energy to organismal maintenance than expected from body mass and taxonomy. Thus, my work lends further support to the idea that human activities promote the homogenisation of the biota, with a set of `winning' species likely to benefit from global changes at the expense of sensitive `losers', putting at risk ecosystem processes sustained by those sensitive species. Overall, my work highlights the necessity of strengthening conservation and mitigation efforts in the face of global human-driven changes. 












